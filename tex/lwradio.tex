% Langwellen-Radio
%
% Dieser Frequenzbereich wird nur aus historischen und Orientierungsgründen noch angezeigt.
% Alle praktisch relevanten Langwellenrundfunksender sind in D inzwischen abgeschaltet.
%

{

	\psset{linewidth=0pt, linestyle=none, fillstyle=solid, fillcolor=BroadcastColour}

	% Langwellenbereich von 148,5 bis 283,5 kHz

	\psframe(1.746,11.55)(9.800,11.75)
	\wrapuparrow{BroadcastColour}{11.65}
	\psframe(0,12.05)(1.107,12.25)
	\wrapdnarrow{BroadcastColour}{12.15}

	\rput(5.77,11.76){\psframebox[framesep=2pt,framearc=0.2]{Langwellenradio}}

	% Switching to lines, we need a different linestyle
	\psset{linewidth=1pt, linestyle=solid}

	%draw baselines on each row for scale
	\psline{|<*-}(1.746,11.55)(9.80,11.55)
	\psline{->|*}(0.00,12.05)(1.107,12.05)

	\psline(2.187,11.550)(2.187,11.575)% 	point=153000.00
	\rput(2.187,11.65){153}
	\psline(2.995,11.550)(2.995,11.575)% 	point=162000.00
	\rput(2.995,11.650){162}
	\psline(3.760,11.550)(3.760,11.575)% 	point=171000.00
	\rput(3.760,11.650){171}
	\psline(4.485,11.550)(4.485,11.575)% 	point=180000.00
	\rput(4.485,11.650){180}
	\psline(5.175,11.550)(5.175,11.575)% 	point=189000.00
	\rput(5.175,11.650){189}
	\psline(5.832,11.550)(5.832,11.575)% 	point=198000.00
	\rput(5.832,11.650){198}
	\psline(6.461,11.550)(6.461,11.575)% 	point=207000.00
	\rput(6.461,11.650){207}
	\psline(7.063,11.550)(7.063,11.575)% 	point=216000.00
	\rput(7.063,11.650){216}
	\psline(7.640,11.550)(7.640,11.575)% 	point=225000.00
	\rput(7.640,11.650){225}
	\psline(8.194,11.550)(8.194,11.575)% 	point=234000.00
	\rput(8.194,11.650){234}
	\psline(8.728,11.550)(8.728,11.575)% 	point=243000.00
	\rput(8.728,11.650){243}
	\psline(9.242,11.550)(9.242,11.575)% 	point=252000.00
	\rput(9.242,11.650){252}
	\psline(9.738,11.550)(9.738,11.575)% 	point=261000.00
	\rput(9.730,11.650){261}

	\psline(0.475,12.050)(0.475,12.075)% 	point=271100.00
	\rput(0.475,12.150){271}
	\psline(0.937,12.050)(0.937,12.075)% 	point=280100.00
	\rput(0.937,12.150){280}
}