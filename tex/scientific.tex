% Scientific spectrum use
{

\definecolor{IonosphericColor}{rgb}{0,0.2,1}
\tiny

%%%%%%%%%%%%%%%%%%%%%%%%%%%%%%%%%%%%%%%%%%%%%%%%%%%%%%%%%%%%%%%%%%%%%%%%%%%%%
% Ionosphärenforschung
%%%%%%%%%%%%%%%%%%%%%%%%%%%%%%%%%%%%%%%%%%%%%%%%%%%%%%%%%%%%%%%%%%%%%%%%%%%%%

\psset{linestyle=solid,linecolor=IonosphericColor,fillstyle=hlines,hatchangle=45,hatchcolor=IonosphericColor,fillcolor=IonosphericColor}

% EISCAT, source: https://eiscat.se/wp-content/uploads/2016/05/Technical.pdf
% Different freq ranges for RX and TX at different sites
% Aggregated to represent typical ranges

% Ionospheric Heater Tromsø 3.85-8 MHz
\psframe(8.589,13.40)(9.8,13.47)\wrapuparrow{IonosphericColor}{13.65}
\psframe(0, 14.01)(9.129,14.075)\rput(4.57,14.05){EISCAT Heater}\wrapdnarrow{IonosphericColor}{14.15}
% VHF Tromsø, Sodankylä and Kiruna % 214.3-234.7 MHz
\psframe(6.616,16.70)(7.901,16.85)\rput(7.26,16.77){EISCAT VHF}
% UHF Longyearbyen 485-515 MHz
\psframe(8.363,17.05)(9.212,17.25)\rput(8.79,17.15){EISCAT Longyearbyen}
% UHF Tromsø 921-933.5 MHz
\psframe(7.631,17.70)(7.821,17.85)\rput(7.71,17.775){\shortstack{EIS\\ CAT}}


%%%%%%%%%%%%%%%%%%%%%%%%%%%%%%%%%%%%%%%%%%%%%%%%%%%%%%%%%%%%%%%%%%%%%%%%%%%%%
% Radioastronomie
%%%%%%%%%%%%%%%%%%%%%%%%%%%%%%%%%%%%%%%%%%%%%%%%%%%%%%%%%%%%%%%%%%%%%%%%%%%%%

% 1420 MHz Wasserstofflinie
%Hydrogen line at 21cm (1.42758313333 GHz)
%Idea from Douglas Scott
%Info from http://www.drao-ofr.hia-iha.nrc-cnrc.gc.ca/outreach/skygazing/1997/sep1197.html
\blip{4.03,18}{H}


%%%%%%%%%%%%%%%%%%%%%%%%%%%%%%%%%%%%%%%%%%%%%%%%%%%%%%%%%%%%%%%%%%%%%%%%%%%%%
% X-Ray Crystallography / Röntgenkristallographie
%%%%%%%%%%%%%%%%%%%%%%%%%%%%%%%%%%%%%%%%%%%%%%%%%%%%%%%%%%%%%%%%%%%%%%%%%%%%%





% für niedrige Kanalkästchen erster Y-Wert -0,05 zweiter -0,15 Textpos. -0,1

}
