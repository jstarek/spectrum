% In Deutschland freigegebene Amateurfunkbänder
%
% Bei Bändern mit sehr unterschiedlichen Nutzungseinschränkungen werden die äußersten Bandgrenzen gezeichnet.
% Die Beschriftung schmaler Bänder ist zur Seite versetzt.
%
% Quelle:  https://de.wikipedia.org/wiki/Amateurfunkband
% Prüfung: https://www.bundesnetzagentur.de/DE/Sachgebiete/Telekommunikation/Unternehmen_Institutionen/Frequenzen/Grundlagen/Frequenzplan/frequenzplan-node.html
% Stand: November 2020

{

\psset{linestyle=none,fillcolor=AmateurRadioColor,fillstyle=solid}

% 2,2 km 135,7-137.8 kHz
\psframe(0.491,11.55)(0.501,11.75)\rput(0.65,11.65){\textcolor{AmateurRadioColor}{2,2\ km}}

% 630 Meter 472-479 kHz
\psframe(8.315,12.05)(8.523,12.25)\rput(8.7,12.15){\textcolor{AmateurRadioColor}{630m}}

% 160 Meter 1.8-2.0MHz
\psframe(7.718,13.05)(9.129,13.25)\rput(8.42,13.15){160\ m}

% 80 Meter 3.5-3.8MHz
\psframe(7.241,13.55)(8.404,13.75)\rput(7.82,13.65){80\ m}

% 60 Meter "Tropenband"
\psframe(3.348,14.05)(3.486,14.25)\rput(3.63,14.15){\textcolor{AmateurRadioColor}{60m}}

% 40 Meter 7.0-7.2MHz
\psframe(7.241,14.05)(7.640,14.25)\rput(7.44,14.15){40\ m}

% 30 Meter 10-10.150MHz
\psframe(2.62,14.55)(2.69,14.75)\rput(2.83,14.65){\textcolor{AmateurRadioColor}{30m}}

% 20 Meter 14.0-14.35MHz
\psframe(7.24,14.55)(7.59,14.75)\rput(7.42,14.65){20\ m}

% 17 Meter 18068-18168 kHz
\psframe(1.05,15.05)(1.13,15.25)\rput(1.27,15.15){\textcolor{AmateurRadioColor}{17m}}

% 15 Meter 21.0-21.45MHz
\psframe(3.17,15.05)(3.47,15.25)\rput(3.32,15.15){15\ m}

% 12 Meter 24890-24990 kHz
\psframe(5.58,15.05)(5.63,15.25)\rput(5.8,15.15){\textcolor{AmateurRadioColor}{12\ m}}

% 10 Meter 28.0-29.7MHz
\psframe(7.241,15.05)(8.075,15.25)\rput(7.66,15.15){10\ m}

% 6 Meter 50-54MHz
\psframe(5.639,15.55)(6.194,15.75)\rput(5.92,15.65){6\ m}

% 2 Meter 144-146MHz
\psframe(0.995,16.55)(1.190,16.75)\rput(1.09,16.65){2m}

% 70 cm 430-440 MHz
\psframe(6.662,17.05)(6.987,17.25)\rput(6.82,17.15){70cm}

% 23 cm  1240-1300 MHz
\psframe(2.035,18.05)(2.703,18.25)\rput(2.37,18.15){23\ cm}

% 13 cm 2320-2450 MHz
\psframe(1.092,18.50)(1.863,18.60)\rput(1.48,18.55){13\ cm}

% 9 cm 3,4-3,475 GHz
\psframe(6.496,18.55)(6.805,18.75)\rput(6.65,18.65){9\ cm}

% 6 cm 5,65-5,85 GHz
\psframe(3.877,19.05)(4.369,19.25)\rput(4.12,19.15){6\ cm}

% 3 cm 10-10,5 GHz
\psframe(2.149,19.55)(2.839,19.75)\rput(2.49,19.65){3\ cm}

% 1,2 cm 24-24,25 GHz
\psframe(4.727,20.05)(4.873,20.25)\rput(4.5,20.15){\textcolor{AmateurRadioColor}{1,2\ cm}}

% 6 mm 47-47,2 GHz
\psframe(4.429,20.55)(4.489,20.75)\rput(4.67,20.65){\textcolor{AmateurRadioColor}{6\ mm}}

% 4 mm 76-81,5 GHz
\psframe(1.424,21.05)(2.412,21.25)\rput(1.92,21.15){4\ mm}

% 2,5 mm 122,25-123 GHz
\psframe(8.144,21.05)(8.231,21.25)\rput(8.47,21.15){\textcolor{AmateurRadioColor}{2,5\ mm}}

% 2 mm 134-141 GHz
\psframe(9.442,21.05)(9.8,21.25)\rput(4.90,21.40){2\ mm}\wrapuparrow{green}{21.2}
\psframe(0.006,21.55)(0.362,21.75)\rput(0.18,21.65){2\ mm}\wrapdnarrow{green}{21.6}

% 1,2 mm 241-250 GHz
\psframe(7.940,21.55)(8.459,21.75)\rput(8.20,21.65){1,2\ mm}

}
