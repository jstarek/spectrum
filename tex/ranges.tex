% Bands of radio waves (encompassing many other bands)
%
%y position = xpos_xvalue/9.8 * 0.5" + nearest xpos_yvalue
%Ex. xpos results for  750e12 937e12   =    (3.72,27.55)(6.62,27.75)   (5.17,27.65)
%    3.72/9.8 * 0.5 + 27.5 = 27.690
%\psframe[fillstyle=solid, fillcolor=BoxColor, linecolor=BoxColor](-0.1,27.690)(9.9,27.838)

%%%%%%%%%%%%%%%%%%%%%%%%%%%%%%%%%%%%%%%%%%%%%%%%%%%%
% Specific commands for frequency chart
%%%%%%%%%%%%%%%%%%%%%%%%%%%%%%%%%%%%%%%%%%%%%%%%%%%%

% \wrapuparrow{color}{Ypos 2 decimal places}
\newcommand{\wrapuparrow}[2]{\rput(9.8,#2){\pscurve[linecolor=#1,fillstyle=none,linewidth=1pt,linestyle=solid]{->}(0,0)(0.1,0.05)(0.02,0.12)}}
\newcommand{\wrapdnarrow}[2]{\rput(0,#2){\pscurve[linecolor=#1,fillstyle=none,linewidth=1pt,linestyle=solid]{->}(0,0)(-0.1,-0.05)(-0.02,-0.12)}}
\newcommand{\wrapbtarrow}[2]{
	\rput(9.8,#2){\pscurve[linecolor=#1,fillstyle=none,linewidth=1pt,linestyle=solid]{->}(0,0)(0.1,0.05)(0.02,0.12)}
	\rput(0,#2){\pscurve[linecolor=#1,fillstyle=none,linewidth=1pt,linestyle=solid]{->}(0,0)(-0.1,-0.05)(-0.02,-0.12)}}


%%%%%%%%%%%%%%%%%%%%%%%%%%%%%%%%%%%%%%%%%%%%%%%%%%%%
% Alternating light / dark grey background in chart
%%%%%%%%%%%%%%%%%%%%%%%%%%%%%%%%%%%%%%%%%%%%%%%%%%%%

\psset{fillstyle=solid,linestyle=none}

% Bottom end of spectrum fade to black
\psframe[fillstyle=solid, fillcolor=Black, linecolor=Black](\EMRPositionC,-0.1)(2.9,0.4)
% conversion of gradient to CMYK does not work for following line, just use solid.
%\psframe[fillstyle=gradient,gradangle=90,gradbegin=Black,gradend=Black,gradmidpoint=1.0,linewidth=0pt,linestyle=none](1.9,-0.1)(2.9,0.4)
\psframe[fillstyle=solid,fillcolor=Black,linewidth=0pt,linestyle=none](1.9,-0.1)(2.9,0.4)

% ULF	Ultra Low Frequency  3 - 30Hz  source  http://www.haarp.alaska.edu/haarp/elf.html
\psframe[fillcolor=LightRange](\EMRPositionC,3.792)(\EMRPositionE,5.453)

% ELF	Extremely Low Frequency  30 - 3kHz
\psframe[fillcolor=DarkRange](\EMRPositionC,5.453)(\EMRPositionE,8.776)

% VLF	Very Low Frequency	100km - 10km	3kHz - 30kHz
\psframe[fillcolor=LightRange](\EMRPositionC,8.776)(\EMRPositionE,10.436)

% LF 	Low Frequency		10km - 1km	30kHz - 300kHz
\psframe[fillcolor=DarkRange](\EMRPositionC,10.436)(\EMRPositionE,12.097)

% MF 	Medium frequency 	1km - 100m	300-3000 kHz
\psframe[fillcolor=LightRange](\EMRPositionC,12.097)(\EMRPositionE,13.758)

% HF 	High frequency 		100m - 10m	3-30 MHz
\psframe[fillcolor=DarkRange](\EMRPositionC,13.758)(\EMRPositionE,15.419)

% VHF	Very High Frequency	10m - 1m	30 - 300 MHz
\psframe[fillcolor=LightRange](\EMRPositionC,15.419)(\EMRPositionE,17.080)

% UHF	Ultra High Frequency	1m - 10cm	300MHz - 3GHz
\psframe[fillcolor=DarkRange](\EMRPositionC,17.080)(\EMRPositionE,18.741)

% SHF Super High Frequency	10cm - 1cm	3GHz - 30GHz
\psframe[fillcolor=LightRange](\EMRPositionC,18.741)(\EMRPositionE,20.402)

% EHF Extremely High Frequency 1cm - 1mm	30GHz - 300GHz
\psframe[fillcolor=DarkRange](\EMRPositionC,20.402)(\EMRPositionE,22.063)

% Near Ultraviolet (NUV)		400-300nm  750-1000THz
\psframe[fillcolor=DarkRange](\EMRPositionC,27.707)(\EMRPositionE,27.916)

% Medium Ultraviolet (MUV)		300-200nm 1000-1500THz
\psframe[fillcolor=LightRange](\EMRPositionC,27.916)(\EMRPositionE,28.207)

% Far Ultraviolet (FUV)		200-100nm 1500-3000THz
\psframe[fillcolor=DarkRange](\EMRPositionC,28.207)(\EMRPositionE,28.705)

% Extreme Ultraviolet (EUV)	200-10nm  1500THz - 30PHz
\psframe[fillcolor=LightRange](\EMRPositionC,28.705)(\EMRPositionE,30.368)


%%%%%%%%%%%%%%%%%%%%%%%%%%%%%%%%%%%%%%%%%%%%%%%%%%%%
% Large left Labels defining the major ranges of EMR
%%%%%%%%%%%%%%%%%%%%%%%%%%%%%%%%%%%%%%%%%%%%%%%%%%%%

{\large\yellow
\psset{linecolor=yellow,linestyle=solid}
\newlength{\LargeLabelPosition}  \setlength{\LargeLabelPosition}{\EMRPosition+0.55in}
\newlength{\LargeLabelRightPosition}  \setlength{\LargeLabelRightPosition}{\EMRPosition+1.2in}

\newlength{\LargeLabelPositionLeft}  \setlength{\LargeLabelPositionLeft}{\LargeLabelPosition-0.2in}
\newlength{\LargeLabelPositionRight}  \setlength{\LargeLabelPositionRight}{\LargeLabelPosition+0.2in}

\uput[180]{90}(\LargeLabelPosition,34.3){GAMMASTRAHLUNG}
\psline[linecolor=LightRange](\EMRPositionC,33.690)(\LargeLabelRightPosition,33.690)
\psline{<->}(\LargeLabelPosition,33.690)(\LargeLabelPosition,35.3)

\psline{<->}(\LargeLabelPosition,33.690)(\LargeLabelPosition,30.368)
\uput[180]{90}(\LargeLabelPosition,32.03){RÖNTGENSTRAHLUNG}

\psline{<->}(\LargeLabelPosition,30.368)(\LargeLabelPosition,27.707)
\uput[180]{90}(\LargeLabelPosition,28.65){ULTRAVIOLETT}

%760-400nm (394-749 THz)
\psline[linecolor=LightRange](\EMRPositionC,27.243)(\LargeLabelRightPosition,27.243)
\psline{<->}(\LargeLabelPosition,27.707)(\LargeLabelPosition,27.243)
\uput[180]{0}(\LargeLabelPosition,27.48){LICHT}

\psline[linecolor=LightRange](\EMRPositionC,23.724)(\LargeLabelRightPosition,23.724)
\psline{<->}(\LargeLabelPosition,27.243)(\LargeLabelPosition,23.724)
\uput[180]{90}(\LargeLabelPosition,25.48){INFRAROT}

%0.3GHz bottom
\psline{<->}(\LargeLabelPosition,23.724)(\LargeLabelPosition,17.080)
\uput[180]{90}(\LargeLabelPosition,19.55){MIKROWELLEN}

\psline{<->}(\LargeLabelPosition,17.080)(\LargeLabelPosition,9.144)
\uput[180]{90}(\LargeLabelPosition,13.11){RADIOWELLEN}

}

\psset{linestyle=solid}% put linestyle back to the way it was


%%%%%%%%%%%%%%%%%%%%%%%%%%%%%%%%%%%%%%%%%%%%%%%%%%%%
% Switch straight / diagonal frequency bands
%%%%%%%%%%%%%%%%%%%%%%%%%%%%%%%%%%%%%%%%%%%%%%%%%%%%

%Change below two lines to this for angled lines
\rput{2.92}{%
\pstilt{87.08}{% angle_for_rput = arctan (0.5"/pspicture_width), pstilt_angle = 90-rput_angle

%change above two lines to this for straight lines
% \rput{0}{
% {



% This is the start of the actual content in the main chart
% Re-sorted to start with the low-lying elements of the chart first:
% backgrounds, frequency lines etc. Above those, coloured
% range backgrounds. Above those, labels for applications.
% On the very top, application-specific band boxes.

%%%%%%%%%%%%%%%%%%%%%%%%%%%%%%%%%%%%%%%%%%%%%%%%%%%%
% White grid lines behind each frequency line
%%%%%%%%%%%%%%%%%%%%%%%%%%%%%%%%%%%%%%%%%%%%%%%%%%%%
\multido{\nYPosition=0.0+0.5,\nYEndPosition=0.3+0.5}{69}{%  The last number here specifies the number of rows
    \psline[linecolor=LineGray]{C-C}(0.000,\nYPosition)(0.000,\nYEndPosition)
    \psline[linecolor=LineGray]{C-C}(0.817,\nYPosition)(0.817,\nYEndPosition)
    \psline[linecolor=LineGray]{C-C}(1.633,\nYPosition)(1.633,\nYEndPosition)
    \psline[linecolor=LineGray]{C-C}(2.450,\nYPosition)(2.450,\nYEndPosition)
    \psline[linecolor=LineGray]{C-C}(3.267,\nYPosition)(3.267,\nYEndPosition)
    \psline[linecolor=LineGray]{C-C}(4.083,\nYPosition)(4.083,\nYEndPosition)
    \psline[linecolor=LineGray]{C-C}(4.900,\nYPosition)(4.900,\nYEndPosition)
    \psline[linecolor=LineGray]{C-C}(5.717,\nYPosition)(5.717,\nYEndPosition)
    \psline[linecolor=LineGray]{C-C}(6.533,\nYPosition)(6.533,\nYEndPosition)
    \psline[linecolor=LineGray]{C-C}(7.350,\nYPosition)(7.350,\nYEndPosition)
    \psline[linecolor=LineGray]{C-C}(8.167,\nYPosition)(8.167,\nYEndPosition)
    \psline[linecolor=LineGray]{C-C}(8.983,\nYPosition)(8.983,\nYEndPosition)
    \psline[linecolor=LineGray]{C-C}(9.800,\nYPosition)(9.800,\nYEndPosition)
    \psline[linecolor=white]{C-C}(0,\nYPosition)(9.8,\nYPosition)%
}
%  \psline[linecolor=LineGray]{C-C}(9.8,-0.5)(9.8,-0.2)% wraparound on first line only

\uput{2pt}[270]( 4.9,-0.1){\psframebox[fillstyle=solid,fillcolor=FColor,linecolor=FColor]{\textcolor{Black}{Frequency}}}
\uput{2pt}[270]( 5.7,-0.1){\psframebox[fillstyle=solid,fillcolor=WColor,linecolor=WColor]{\textcolor{Black}{Wavelength}}}
\uput{2pt}[270]( 6.5,-0.1){\psframebox[fillstyle=solid,fillcolor=EColor,linecolor=EColor]{\textcolor{Black}{Energy}}}

\newcounter{StartHeight}	\setcounter{StartHeight}{3}
\newcounter{EndHeight}	\setcounter{EndHeight}{4}

%From Doug Welch: Crosshatch style for non-visible colors
\psset{fillstyle=crosshatch,linewidth=0pt,linestyle=none, hatchwidth=2pt, hatchsep=1.5pt}


%%%%%%%%%%%%%%%%%%%%%%%%%%%%%%%%%%%%%%%%%%%%%%%%%%%%
% Frequency, energy and wavelength labels
%%%%%%%%%%%%%%%%%%%%%%%%%%%%%%%%%%%%%%%%%%%%%%%%%%%%

% pull in precompiled labels
\input{tex/numbers.tex}


%%%%%%%%%%%%%%%%%%%%%%%%%%%%%%%%%%%%%%%%%%%%%%%%%%%%
% Cross-hatched range backgrounds
%%%%%%%%%%%%%%%%%%%%%%%%%%%%%%%%%%%%%%%%%%%%%%%%%%%%

% Brain waves
% Delta waves 0.1 3Hz
\definecolor{Fill}{rgb}{1,0.9,0.9}\psset{hatchcolor=Fill}
\psframe(6.55,1.05)(9.80,1.25)\wrapuparrow{Fill}{1.15}
\psframe(0.00,1.55)(9.80,1.75)\wrapbtarrow{Fill}{1.65}
\psframe(0.00,2.05)(9.80,2.25)\wrapbtarrow{Fill}{2.15}
\psframe(0.00,2.55)(9.80,2.75)\wrapbtarrow{Fill}{2.65}
\psframe(0.00,3.05)(9.80,3.25)\wrapbtarrow{Fill}{3.15}
\psframe(0.00,3.55)(5.73,3.75)\wrapdnarrow{Fill}{3.65}
\rput(6.55,1.15){\psframebox[framearc=0.25,fillstyle=solid, fillcolor=Fill,linecolor=Black,framesep=1pt]{0.1Hz}}
\rput(4.9,2.65){\psframebox[fillstyle=solid,fillcolor=Fill,framesep=1pt]{\boldmath$\delta$ (Delta-Wellen)}}
{\psset{linecolor=yellow,linestyle=solid,linewidth=1pt}
\psline{<-}(6.55,1.05)(9.80,1.05)
\psline{-}(0,1.55)(9.80,1.55)
\psline{-}(0,2.05)(9.80,2.05)
\psline{-}(0,2.55)(9.80,2.55)
\psline{-}(0,3.05)(9.80,3.05)
\psline{->}(0,3.55)(5.73,3.55)
}

% Theta 3-8 Hz
\definecolor{Fill}{rgb}{1,0.8,0.8}\psset{hatchcolor=Fill}
\psframe(5.73,3.55)(9.8,3.75)\wrapuparrow{Fill}{3.65}
\psframe(0.00,4.05)(9.8,4.25)\wrapdnarrow{Fill}{4.15}
\rput(5.73,3.65){\psframebox[framearc=0.25,fillstyle=solid, fillcolor=Fill,linecolor=Black,framesep=1pt]{3Hz}}
\rput(4.9,4.15){\psframebox[fillstyle=solid,fillcolor=Fill,framesep=1pt]{\boldmath$\theta$ (Theta-Wellen)}}
\rput(9.8,4.15){\psframebox[framearc=0.25,fillstyle=solid, fillcolor=Fill,linecolor=Black]{8Hz}}
{\psset{linecolor=yellow,linestyle=solid,linewidth=1pt}
\psline{<-}(5.73,3.55)(9.8,3.55)
\psline{->}(0.00,4.05)(9.8,4.05)
}

% Alpha 8-12Hz
\definecolor{Fill}{rgb}{1,0.7,0.7}\psset{hatchcolor=Fill}
\psframe(0.00,4.55)(5.73,4.75)
\rput(0,4.65){\psframebox[framearc=0.25,fillstyle=solid, fillcolor=Fill,linecolor=Black,framesep=1pt]{8Hz}}
\rput(2.87,4.65){\psframebox[fillstyle=solid,fillcolor=Fill,framesep=1pt]{\boldmath$\alpha$ (Alpha-Wellen)}}
\psline[linecolor=yellow,linestyle=solid,linewidth=1pt]{<->}(0.00,4.55)(5.73,4.55)

% Low Beta 12-15
\definecolor{Fill}{rgb}{1,0.6,0.6}\psset{hatchcolor=Fill}
\psframe(5.73,4.55)(8.89,4.75)
\rput(5.73,4.65){\psframebox[framearc=0.25,fillstyle=solid, fillcolor=Fill,linecolor=Black,framesep=1pt]{12Hz}}
\rput(6.82,4.65){\psframebox[fillstyle=solid,fillcolor=Fill,framesep=1pt]{\boldmath$\beta$ (Low Beta brain waves)}}
\psline[linecolor=yellow,linestyle=solid,linewidth=1pt]{<->}(5.73,4.55)(8.89,4.55)

% Mid Beta 15-18
\definecolor{Fill}{rgb}{1,0.5,0.5}\psset{hatchcolor=Fill}
\psframe(8.89,4.55)(9.8,4.75)\wrapuparrow{Fill}{4.65}
\psframe(0.00,5.05)(1.67,5.25)\wrapdnarrow{Fill}{5.15}
\rput(8.89,4.65){\psframebox[framearc=0.25,fillstyle=solid, fillcolor=Fill,linecolor=Black,framesep=1pt]{15Hz}}
\rput(0.7,5.15){\psframebox[fillstyle=solid,fillcolor=Fill,framesep=1pt]{\boldmath$\beta$ (Mid Beta brain waves)}}
{\psset{linecolor=yellow,linestyle=solid,linewidth=1pt}
\psline{<-}(8.89,4.55)(9.8,4.55)
\psline{->}(0.00,5.05)(1.67,5.05)
}

% High Beta 18-30Hz
\definecolor{Fill}{rgb}{1,0.4,0.4}\psset{hatchcolor=Fill}
\psframe(1.67,5.05)(8.89,5.25)
\rput(1.67,5.15){\psframebox[framearc=0.25,fillstyle=solid, fillcolor=Fill,linecolor=Black,framesep=1pt]{\textcolor{white}{18Hz}}}
\rput(5.28,5.15){\psframebox[fillstyle=solid,fillcolor=Fill,framesep=1pt]{\boldmath$\beta$ (High Beta brain waves)}}
\psline[linecolor=yellow,linestyle=solid,linewidth=1pt]{<->}(1.67,5.05)(8.89,5.05)

% Gamma 30Hz
\definecolor{Fill}{rgb}{1,0.3,0.3}\psset{hatchcolor=Fill}
\psframe(8.89,5.05)(9.8,5.25)\wrapuparrow{Fill}{5.15}
\psframe(0.00,5.55)(3.0,5.75)\wrapdnarrow{Fill}{5.65}
\psframe[fillstyle=gradient,gradangle=90,gradbegin=Fill,gradend=DarkRange,gradmidpoint=1.0,linewidth=0pt,linestyle=none](3.0,5.55)(4.0,5.75)
\rput(8.89,5.15){\psframebox[framearc=0.25,fillstyle=solid, fillcolor=Fill,linecolor=Black,framesep=1pt]{\textcolor{white}{30Hz}}}
\rput(1.5,5.65){\psframebox[fillstyle=solid,fillcolor=Fill,framesep=1pt]{\boldmath$\gamma$ (Gamma brain waves)}}
{\psset{linecolor=yellow,linestyle=solid,linewidth=1pt}
\psline{<-}(8.89,5.05)(9.8,5.05)
\psline{->}(0.00,5.55)(4.0,5.55)
}

% Marine Radio 235-325kHz
%\definecolor{Fill}{rgb}{0.37,0.86,0.86}
%\psframe[fillstyle=solid, fillcolor=Fill,linewidth=0pt,linestyle=none](8.25,11.60)(9.80,11.75)\wrapuparrow{Fill}{11.65}
%\psframe[fillstyle=solid, fillcolor=Fill,linewidth=0pt,linestyle=none](0.00,12.10)(3.04,12.25)\wrapdnarrow{Fill}{12.15}
%\psline{|<*-}(8.25,11.60)(9.8,11.60)
%\psline{->|*}(0.00,12.10)(3.04,12.10)
%\uput{1pt}[10](8.25,11.65){235kHz}
%\rput(1,12.15){Marine Radio}
%\uput{1pt}[170](3.04,12.15){325kHz}

% Microwave bands (notice even octave spacing on most bands)
\psset{fillstyle=crosshatch,linewidth=0pt,linestyle=none}

% Microwave P-band (0.3-1 GHz)
% Commented out due to collisions with cellphone bands
%\definecolor{Fill}{rgb}{0.1,0.1,0.67}\psset{hatchcolor=Fill}
%\psframe(1.57,17.0)(9.8,17.1)\wrapuparrow{Fill}{17.05}
%\psframe(0.0,17.5)(8.79,17.6)\wrapdnarrow{Fill}{17.55}
%\rput(1,17.55){\psframebox[fillstyle=solid,fillcolor=Fill,framesep=1pt]{\textcolor{white}{\tiny P-Band}}}

% Microwave L-band (1-2 GHz)
\definecolor{Fill}{rgb}{0.2,0.1,0.67}\psset{hatchcolor=Fill}
\psframe(8.79,17.55)(9.8,17.75)\wrapuparrow{Fill}{17.65}
\psframe(0.0,18.05)(8.79,18.25)\wrapdnarrow{Fill}{18.15}
\psdots[linewidth=1.2pt,linecolor=white,linestyle=none, fillcolor=white, dotstyle=triangle*](8.79,17.52)
\rput(8.79,17.75){\textcolor{white}{1GHz}}
\rput(4.715,18.14){\psframebox[fillstyle=solid,fillcolor=Fill,framesep=2pt]{\textcolor{white}{L-Band}}}

% Microwave S-band (2-4 GHz)
\definecolor{Fill}{rgb}{0.3,0.1,0.67}\psset{hatchcolor=Fill}
\psframe(8.79,18.05)(9.8,18.25)\wrapuparrow{Fill}{18.15}
\psframe(0.0,18.55)(8.79,18.75)\wrapdnarrow{Fill}{18.65}
\psdots[linewidth=1.2pt,linecolor=white,linestyle=none, fillcolor=white, dotstyle=triangle*](8.79,18.02)
\rput(8.79,18.14){\textcolor{white}{2GHz}}
\rput(5.5,18.65){\psframebox[fillstyle=solid,fillcolor=Fill,framesep=2pt]{\textcolor{white}{S-band}}}


{
	% Microwave oven 2.45GHz
	\rput[l](1.96,18.62){\textcolor{yellow}{2.45GHz Mikrowellenherd}}
	\psline{-*}(1.92,18.62)(1.86,18.62)(1.86,18.5)
}

% Microwave C-band (4-8 GHz)
\definecolor{Fill}{rgb}{0.4,0.1,0.67}\psset{hatchcolor=Fill}
\psframe(8.79,18.55)(9.8,18.75)\wrapuparrow{Fill}{18.65}
\psframe(0.0,19.05)(8.79,19.25)\wrapdnarrow{Fill}{19.15}
\psdots[linewidth=1.2pt,linecolor=white,linestyle=none, fillcolor=white, dotstyle=triangle*](8.79,18.52)
\rput(8.79,18.77){\textcolor{white}{4GHz}}
\rput(5.5,19.15){\psframebox[fillstyle=solid,fillcolor=Fill,framesep=2pt]{\textcolor{white}{C-band}}}

% Microwave Ka-band (27.25 - 36 GHz)
\definecolor{Fill}{rgb}{0.8,0.1,0.67}\psset{hatchcolor=Fill}
\psframe(6.52,20.05)(9.8,20.25)\wrapuparrow{Fill}{20.15}
\psframe(0.0,20.55)(0.66,20.75)\wrapdnarrow{Fill}{20.65}
\psline[linecolor=Black,linewidth=1pt,linestyle=solid](6.52,20.05)(6.52,20.25)
\psdots[linewidth=1.2pt,linecolor=white,linestyle=none, fillcolor=white, dotstyle=triangle*](6.52,20.02)
\rput(6.52,20.15){\textcolor{white}{27.25GHz}}
\rput(8,20.15){\psframebox[fillstyle=solid,fillcolor=Fill,framesep=2pt]{\textcolor{white}{Ka-Band}}}

%from http://www.jneuhaus.com/fccindex/letter.html
%Q  36 - 46 GHz
\definecolor{Fill}{rgb}{0.9,0.1,0.67}\psset{hatchcolor=Fill}
\psframe(0.66,20.55)(4.12,20.75)
\psline[linecolor=Black,linewidth=1pt,linestyle=solid](0.66,20.55)(0.66,20.75)
\psdots[linewidth=1.2pt,linecolor=white,linestyle=none, fillcolor=white, dotstyle=triangle*](0.66,20.52)
\rput(0.66,20.65){\textcolor{white}{36GHz}}
\rput(2.39,20.65){\psframebox[fillstyle=solid,fillcolor=Fill,framesep=2pt]{\textcolor{white}{Q-Band}}}

%V  46 - 56 GHz
\definecolor{Fill}{rgb}{0.9,0.1,0.67}\psset{hatchcolor=Fill}
\psframe(4.12,20.55)(6.91,20.75)
\psline[linecolor=Black,linewidth=1pt,linestyle=solid](4.12,20.55)(4.12,20.75)
\psdots[linewidth=1.2pt,linecolor=white,linestyle=none, fillcolor=white, dotstyle=triangle*](4.12,20.52)
\rput(4.12,20.65){\textcolor{white}{46GHz}}
\rput(5.52,20.65){\psframebox[fillstyle=solid,fillcolor=Fill,framesep=2pt]{\textcolor{white}{Microwave V-band}}}

%W  56 - 100 GHz
\definecolor{Fill}{rgb}{0.9,0.1,0.67}\psset{hatchcolor=Fill}
\psframe(6.91,20.55)(9.80,20.75)\wrapuparrow{Fill}{20.65}
\psframe(0.00,21.05)(5.30,21.25)\wrapdnarrow{Fill}{21.15}
\psline[linecolor=Black,linewidth=1pt,linestyle=solid](6.91,20.55)(6.91,20.75)
\psdots[linewidth=1.2pt,linecolor=white,linestyle=none, fillcolor=white, dotstyle=triangle*](6.91,20.52)
\rput(6.91,20.65){\textcolor{white}{56GHz}}
\rput(2.65,21.15){\psframebox[fillstyle=solid,fillcolor=Fill,framesep=2pt]{\textcolor{white}{Microwave W-band}}}

%remaining undefined microwave band 100 - 3000GHz
{
\definecolor{Fill}{rgb}{1.0,0.1,0.67}\psset{hatchcolor=Fill,hatchwidth=1pt,hatchsep=3pt}
\psframe(5.30,21.05)(9.80,21.25)\wrapuparrow{Fill}{21.15}
\psframe(0.00,21.55)(9.80,21.75)\wrapbtarrow{Fill}{21.65}
\psframe(0.00,22.05)(9.80,22.25)\wrapuparrow{Fill}{22.15}
\psframe(0.00,22.55)(9.80,22.75)\wrapbtarrow{Fill}{22.65}
\psframe(0.00,23.05)(9.80,23.25)\wrapbtarrow{Fill}{23.15}
\psframe(0.00,23.55)(4.39,23.75)\wrapdnarrow{Fill}{23.65}
\psline[linecolor=Black,linewidth=1pt,linestyle=solid](5.30,21.05)(5.30,21.25)
\psdots[linewidth=1.2pt,linecolor=white,linestyle=none, fillcolor=white, dotstyle=triangle*](5.30,21.02)
\rput(5.30,21.15){\textcolor{white}{100GHz}}
}

% Far Infrared 100-30um	3-10 THz
\definecolor{Fill}{rgb}{0.9,0.0,0.0}\psset{hatchcolor=Fill}
\psframe(4.39,23.55)(9.80,23.75)\wrapuparrow{Fill}{23.65}
\psframe(0.00,24.05)(9.80,24.25)\wrapbtarrow{Fill}{24.15}
\psframe(0.00,24.55)(1.81,24.75)\wrapdnarrow{Fill}{24.65}
\psdots[linewidth=1.2pt,linecolor=white,linestyle=none, fillcolor=white, dotstyle=triangle*](4.39,23.52)
\rput(4.39,23.70){\textcolor{white}{\si{100\micro\meter}}}
\rput(4.39,23.60){\textcolor{white}{3THz}}
\rput(4.9,24.15){\psframebox[fillstyle=solid,fillcolor=Fill,framesep=2pt]{\textcolor{white}{Far Infrared}}}


% Thermal Infrared 30-3m	10-100 THz
\definecolor{Fill}{rgb}{0.8,0,0}\psset{hatchcolor=Fill}
\psframe(1.81,24.55)(9.80,24.75)\wrapuparrow{Fill}{24.65}
\psframe(0.00,25.05)(9.80,25.25)\wrapbtarrow{Fill}{25.15}
\psframe(0.00,25.55)(9.80,25.75)\wrapbtarrow{Fill}{25.65}
\psframe(0.00,26.05)(4.97,26.25)\wrapdnarrow{Fill}{26.15}
\psdots[linewidth=1.2pt,linecolor=white,linestyle=none, fillcolor=white, dotstyle=triangle*](1.81,24.52)
\rput(1.81,24.65){\textcolor{white}{\si{30\micro\meter}}}
\rput(4.9,25.65){\psframebox[fillstyle=solid,fillcolor=Fill,framesep=2pt]{\textcolor{white}{Thermal Infrared}}}

% Near Infrared 3um-760nm 	100-394 THz
\definecolor{Fill}{rgb}{0.7,0,0}\psset{hatchcolor=Fill}
\psframe(4.97,26.05)(9.80,26.25)\wrapuparrow{Fill}{26.15}
\psframe(0.00,26.55)(9.80,26.75)\wrapbtarrow{Fill}{26.65}
\psframe(0.00,27.05)(4.75,27.25)\wrapdnarrow{Fill}{27.15}
\psdots[linewidth=1.2pt,linecolor=white,linestyle=none, fillcolor=white, dotstyle=triangle*](4.97,26.02)
\rput(4.97,26.15){\textcolor{white}{\si{3\micro\meter}}}
\rput(5.3,26.65){\psframebox[fillstyle=solid,fillcolor=Fill,framesep=2pt]{\textcolor{white}{Near Infrared}}}

  % The hsb color model starts and stops at Red (Hue = 0 to 1)
  %   which is different than the visible spectrum which starts at Red and ends at Violet.
  % Therefore, the range must be divided into pieces.
  % Some color fading between violet and red must be ignored
  %
  % Visible spectrum 394-749THz (two rows)
  % Go from Red to Green (first row of visible spectrum)
%  \multido{%
%     \nStartColor=0.00+0.03,
%     \nEndColor=0.03+0.03,
%     \nLeftSide=4.750+0.505,
%     \nRightSide=5.255+0.505}{10}{%
%  \definecolor{StartColor}{hsb}{\nStartColor,1,1}
%  \definecolor{EndColor}{hsb}{\nEndColor,1,1}
%  \psframe[fillstyle=gradient,
%  		gradangle=90,
%		gradbegin=StartColor,
%		gradend=EndColor,
%		gradmidpoint=1.0,
%		linewidth=0pt,
%		linestyle=none]
%		(\nLeftSide,27.05)(\nRightSide,27.25)}
%
%  % Go from Green to Violet (second row of visible spectrum)
%  \multido{%
%     \nStartColor=0.300+0.055,
%     \nEndColor=0.355+0.055,
%     \nLeftSide=0.00+0.404,
%     \nRightSide=0.404+0.404}{10}{%
%  \definecolor{StartColor}{hsb}{\nStartColor,1,1}
%  \definecolor{EndColor}{hsb}{\nEndColor,1,1}
%  \psframe[fillstyle=gradient,
%  		gradangle=90,
%		gradbegin=StartColor,
%		gradend=EndColor,
%		gradmidpoint=1.0,
%		linewidth=0pt,
%		linestyle=none]
%		(\nLeftSide,27.55)(\nRightSide,27.75)}
%  {\textcolor{WColor}{
%  \rput(4.77,27.2){7600\angstrom}
%  \rput(4.77,27.1){760nm}
%  }}


{
\psset{gradlines=100,fillstyle=gradient,gradangle=90,gradmidpoint=1.0,linewidth=0pt,linestyle=none}
\definecolor{StartColor}{hsb}{.0,1,1} \definecolor{EndColor}{hsb}{.1,1,1} \psframe[gradbegin=StartColor,gradend=EndColor](4.77,27.05)(7.09,27.25) %Red to Orange
\definecolor{StartColor}{hsb}{.1,1,1} \definecolor{EndColor}{hsb}{.2,1,1} \psframe[gradbegin=StartColor,gradend=EndColor](7.09,27.05)(9.09,27.25) %Orange to Yellow
\definecolor{StartColor}{hsb}{.2,1,1} \definecolor{EndColor}{hsb}{.38,1,1}\psframe[gradbegin=StartColor,gradend=EndColor](9.09,27.05)(9.80,27.25) %Yellow to Green far right
\wrapuparrow{EndColor}{27.15}
\definecolor{StartColor}{hsb}{.38,1,1}\definecolor{EndColor}{hsb}{.4,1,1}
\wrapdnarrow{StartColor}{27.65}
\psframe[gradbegin=StartColor,gradend=EndColor](0.00,27.55)(0.61,27.75) %Yellow to Green far left
\definecolor{StartColor}{hsb}{.4,1,1} \definecolor{EndColor}{hsb}{.5,1,1} \psframe[gradbegin=StartColor,gradend=EndColor](0.61,27.55)(2.23,27.75) %Green to Blue
\definecolor{StartColor}{hsb}{.5,1,1} \definecolor{EndColor}{hsb}{.8,1,1} \psframe[gradbegin=StartColor,gradend=EndColor](2.23,27.55)(4.05,27.75) %Blue to Violet
}

%  Red 		760-620 nm (690nm, 434.5THz)	394 - 483 THz
%  Orange 	620-570 nm (645nm, 464.8THz)	483 - 526 THz
%  Yellow 	570-550 nm (560nm, 535.3THz)	526 - 545 THz
%  Green 	550-470 nm (510nm, 587.8THz)	545 - 638 THz
%  Blue 	470-440 nm (455nm, 658.9THz)	638 - 681 THz
%  Violet 	440-380 nm (410nm, 731.2THz)	681 - 789 THz

%midpoints
%(4.77
%(6.14,27.05)	red
%(7.09,27.25)	orange
%(9.09,27.05)	yellow
%(0.61,27.75)	green
%(2.23,27.55)	blue
%(3.70,27.75)	violet
%4.05

  %graduations in the visible range from 760nm to 400nm in steps of 10nm
  \psdots[linewidth=1.2pt,linecolor=white,linestyle=none, fillcolor=white, dotstyle=triangle*]
	(4.77,27.02)(4.96,27.02)(5.15,27.02)(5.34,27.02)(5.54,27.02)(5.73,27.02)(5.93,27.02)(6.14,27.02)
	(6.34,27.02)(6.55,27.02)(6.77,27.02)(6.98,27.02)(7.20,27.02)(7.42,27.02)(7.65,27.02)(7.88,27.02)
	(8.11,27.02)(8.35,27.02)(8.59,27.02)(8.84,27.02)(9.09,27.02)(9.34,27.02)(9.60,27.02)(0.07,27.52)
	(0.34,27.52)(0.61,27.52)(0.89,27.52)(1.18,27.52)(1.47,27.52)(1.77,27.52)(2.07,27.52)(2.38,27.52)
	(2.70,27.52)(2.70,27.52)(3.02,27.52)(3.36,27.52)(3.70,27.52)(4.05,27.52)

  \rput(8.37,27.15){Sichtbares Licht}
  \rput(2,27.65){Sichtbares Licht}



  %Ultraviolet section
  \definecolor{UVAColor}{rgb}{0.4,0.0,0.6}
  \definecolor{UVBColor}{rgb}{0.6,0.0,0.8}

  % UV-A 400-320 nm	749 - 937 THz
  \psframe[hatchcolor=UVAColor](4.04,27.55)(7.20,27.75)

  %Gradient between contentious range 320nm to 315nm
  \psframe[fillstyle=gradient,gradangle=90,gradbegin=UVAColor,gradend=UVBColor,
		gradmidpoint=1.0,linewidth=0pt,linestyle=none](7.20,27.55)(7.42,27.75)

  % UV-B 315 - 280 nm	937 - 1071 THz
  \psframe[hatchcolor=UVBColor](7.42,27.55)(9.09,27.75)

  \psline[linestyle=solid,linecolor=Black,linewidth=1pt](4.04,27.55)(4.04,27.75)% Vertical axes line
  \psline[linestyle=solid,linecolor=Black,linewidth=1pt](6.34,27.55)(6.34,27.65)% Vertical axes line
  \psline[linestyle=solid,linecolor=Black,linewidth=1pt](7.20,27.55)(7.20,27.75)% Vertical axes line
  \psline[linestyle=solid,linecolor=Black,linewidth=1pt](7.42,27.55)(7.42,27.75)% Vertical axes line

  \psline[linestyle=solid,linecolor=red,linewidth=1pt]{<->}(4.04,27.69)(7.20,27.69)% Arrow dimension indicator UVA
  \psline[linestyle=solid,linecolor=red,linewidth=1pt]{<->}(4.04,27.61)(6.34,27.61)% Arrow dimension indicator UVA1
  \psline[linestyle=solid,linecolor=red,linewidth=1pt]{<->}(6.34,27.61)(7.20,27.61)% Arrow dimension indicator UVA2
  \psline[linestyle=solid,linecolor=red,linewidth=1pt]{->}(7.10,27.69)(7.42,27.69)% Arrow dimension indicator end of UVA (contentious zone)

  % UVA1 400-340nm  749THz-881THz
  % UVA2 340-315nm  881THz-937THz
  {\textcolor{WColor}{
    \rput(4.04,27.79){400nm}
    \psset{fillcolor=UVAColor, fillstyle=solid, framesep=1pt}
    \rput(5.985,27.68){\psframebox{UV-A}}
    \rput(5.195,27.61){\psframebox{UV-A1}}
    \rput(6.77,27.61){\psframebox{UV-A2}}
    \rput(6.34,27.79){340nm}
    \rput(7.31,27.79){320 315nm}
    \rput(8.15,27.65){UV-B}
  }}

% UV-C 280-100 nm	1071 - 3000 THz
\definecolor{Fill}{rgb}{0.8,0.0,1}\psset{hatchcolor=Fill}
\psframe(9.09,27.55)(9.80,27.75)\wrapuparrow{Fill}{27.65}
\psframe(0.00,28.05)(9.80,28.25)\wrapbtarrow{Fill}{28.15}
\psframe(0.00,28.55)(4.06,28.75)\wrapdnarrow{Fill}{28.65}
{\textcolor{WColor}{
\psline[linestyle=solid,linecolor=Black,linewidth=1pt](9.09,27.55)(9.09,27.75)
\rput(9.09,27.65){280nm}
\rput(2,28.15){UV-C}
}}

% Extreme Ultraviolet	100- 10 nm	3 - 30 PHz
\definecolor{Fill}{rgb}{.9,0,1.0}\psset{hatchcolor=Fill}
\psframe(4.06,28.55)(9.80,28.75)\wrapuparrow{Fill}{28.65}
\psframe(0.00,29.05)(9.80,29.25)\wrapbtarrow{Fill}{29.15}
\psframe(0.00,29.55)(9.80,29.75)\wrapbtarrow{Fill}{29.65}
\psframe(0.00,30.05)(7.21,30.25)\wrapdnarrow{Fill}{30.15}
{\textcolor{WColor}{
\psline[linestyle=solid,linecolor=Black,linewidth=1pt](4.06,28.55)(4.06,28.75)
%\rput(4.90,29.2){EUV}
%\rput(4.90,29.1){(Extreme Ultraviolet)}
\rput(4.90,29.15){\psframebox[fillstyle=solid,fillcolor=Fill,framesep=2pt]{EUV (Extreme Ultraviolet)}}
}}
%\rput(4.06,28.65){100nm}
\rput[B](4.06,28.5){\psframebox[fillstyle=solid,fillcolor=white,framesep=1pt]{100nm}}

% Soft XRay 	10-1nm		 30 - 300  PHz
\definecolor{Fill}{rgb}{1,0.5,0}\psset{hatchcolor=Fill}
\definecolor{SoftXrayColor}{rgb}{1,0.5,0}
\psframe(7.21,30.05)(9.80,30.25)\wrapuparrow{Fill}{30.15}
\psframe(0.00,30.55)(9.80,30.75)\wrapbtarrow{Fill}{30.65}
\psframe(0.00,31.05)(9.80,31.25)\wrapbtarrow{Fill}{31.15}
\psframe(0.00,31.55)(9.80,31.75)\wrapbtarrow{Fill}{31.65}
\psframe(0.00,32.05)(0.57,32.25)\wrapdnarrow{Fill}{32.15}
%
%\psframe[fillstyle=gradient,gradangle=90,gradbegin=EUVColor,gradend=SoftXrayColor,
%		gradmidpoint=1.0,linewidth=0pt,linestyle=none](6.91,30.05)(7.51,30.25)
%
{\textcolor{WColor}{
\psline[linestyle=solid,linecolor=Black,linewidth=1pt](7.21,30.05)(7.21,30.25)
%\rput(7.21,30.15){10nm}
}}
\rput[B](7.21,30){\psframebox[fillstyle=solid,fillcolor=white,framesep=1pt]{10nm}}
\rput(4.90,31.15){\psframebox[fillstyle=solid,fillcolor=Fill,framesep=2pt]{\textcolor{Black}{Soft XRay}}}

% Hard XRay 	0.1-1 nm	 .3 - 3    EHz
\definecolor{Fill}{rgb}{1,0.9,0}\psset{hatchcolor=Fill}
\definecolor{HardXrayColor}{rgb}{1,0.9,0}
\psframe(0.57,32.05)(9.80,32.25)\wrapuparrow{Fill}{32.15}
\psframe(0.00,32.55)(9.80,32.75)\wrapbtarrow{Fill}{32.65}
\psframe(0.00,33.05)(9.80,33.25)\wrapbtarrow{Fill}{33.15}
\psframe(0.00,33.55)(3.72,33.75)\wrapdnarrow{Fill}{33.65}
%
%  \psframe[fillstyle=gradient,gradangle=90,gradbegin=SoftXrayColor,gradend=HardXrayColor,gradmidpoint=1.0,linewidth=0pt,linestyle=none](0.02,32.05)(1.02,32.25)
%
\psline[linestyle=solid,linecolor=Black,linewidth=1pt](0.57,32)(0.57,32.25)
\rput[B](0.57,32){\psframebox[fillstyle=solid,fillcolor=white,framesep=1pt]{1nm}}
\rput(4.90,32.65){\psframebox[fillstyle=solid,fillcolor=Fill,framesep=2pt]{\textcolor{Black}{Hard XRay}}}




% Gamma Ray 	0.1nm-infinity	 3EHz-infinity
%
% Put black box to cover excess number labels
\psframe[fillstyle=solid, fillcolor=Black,linewidth=0pt,linestyle=none](6,33.8)(10.5,34.5)
%
\definecolor{GammRayColor}{rgb}{.7,1,.7}\psset{hatchcolor=GammRayColor}
\psframe(3.72,33.55)(9.80,33.75)\wrapuparrow{GammRayColor}{33.65}
\psframe(0.00,34.05)(6.00,34.25)\wrapdnarrow{GammRayColor}{34.15}
%
%\psframe[fillstyle=gradient,gradangle=90,gradbegin=HardXrayColor,gradend=GammRayColor,gradmidpoint=1.0,linewidth=0pt,linestyle=none](3.42,33.55)(4.02,33.75)
\psframe[fillstyle=gradient,gradangle=90,gradbegin=GammRayColor,gradend=Black,gradmidpoint=1.0,linewidth=0pt,linestyle=none](6.00,34.05)(7,34.25)
%
\psline[linestyle=solid,linecolor=Black,linewidth=1pt](3.72,33.5)(3.72,33.75)
\rput[B](3.72,33.5){\psframebox[fillstyle=solid,fillcolor=white,framesep=1pt]{0.1nm}}
\rput(6,33.65){\psframebox[fillstyle=solid,fillcolor=GammRayColor,framesep=2pt]{\textcolor{Black}{Gamma Ray}}}
\psline[linecolor=white,linestyle=solid,linewidth=1pt]{->}(6.9,34.152)(7,34.152)
\psframe[fillstyle=gradient,gradangle=90,gradbegin=GammRayColor,gradend=white,gradmidpoint=1.0,linewidth=0pt,linestyle=none](6,34.143)(6.97,34.157)
\uput{2pt}[0](7.07,34.02){\textcolor{EColor}{$\infty$eV}}
\uput{2pt}[0](7.07,34.15){\textcolor{WColor}{$\frac{1}{\infty}\si{\meter}$}}
\uput{2pt}[0](7.07,34.28){\textcolor{FColor}{$\infty$Hz}}
\psframe[fillstyle=gradient,gradangle=90,gradbegin=white,gradend=Black,gradmidpoint=1.0,linewidth=0pt,linestyle=none](5.98,33.991)(7,34.002)


% Cover bottom end range to identify infinity
%
% Put black box to cover unwanted lines and number labels
\psframe[fillstyle=solid,fillcolor=RGBBlackBegin,linewidth=0pt,linestyle=none](-0.22,-.05)(1.9,0.31)
%The following does not work with CMYK because gradient are probably RGB
%\psframe[fillstyle=gradient,gradangle=90,gradbegin=Black,gradend=Black,gradmidpoint=1.0,linewidth=0pt,linestyle=none](1.9,-0.1)(2.9,0.31)
\psframe[fillstyle=solid,fillcolor=RGBBlackBegin,linewidth=0pt,linestyle=none](1.9,-0.1)(2.9,0.31)
%
%Arrow
\psline[linecolor=white,linestyle=solid,linewidth=1pt]{<-}(1.9,0.152)(2.1,0.152)
\psframe[fillstyle=gradient,gradangle=90,gradbegin=white,gradend=DarkRange,gradmidpoint=1.0,linewidth=0pt,linestyle=none](2,0.146)(2.9,0.157)
%
\uput{2pt}[180](1.9,0.02){\textcolor{EColor}{$\frac{1}{\infty}$eV}}
\uput{2pt}[180](1.9,0.15){\textcolor{WColor}{$\infty\si{\meter}$}}
\uput{2pt}[180](1.9,0.28){\textcolor{FColor}{$\frac{1}{\infty}$Hz}}
\psframe[fillstyle=gradient,gradangle=90,gradbegin=Black,gradend=white,gradmidpoint=1.0,linewidth=0pt,linestyle=none](1.9,-.005)(2.9,0.006)



%%%%%%%%%%%%%%%%%%%%%%%%%%%%%%%%%%%%%%%%%%%%%%%%%%%%
% General application areas, yellow, from LW to VHF
%%%%%%%%%%%%%%%%%%%%%%%%%%%%%%%%%%%%%%%%%%%%%%%%%%%%

%For some reason this line cannot be preceded by an \input{} statement
{\yellow

  % GWEN 150 to 175 kHz from http://www.dxing.com/lw.htm
  \psset{fillstyle=solid,linecolor=yellow,linewidth=1pt,linestyle=solid,framesep=0pt,fillcolor=yellow}
  \psline{|<*->|}(1.91,11.55)(4.09,11.55)\rput(3.00,11.55){\psframebox{\textcolor{Black}{Ground Wave Emergency Network}}}

  %155-281kHz from http://www.dxing.com/lw.htm
  \psline{|<*-}(2.37,11.75)(9.8,11.75)\wrapuparrow{yellow}{11.75}\rput(6.09,11.75){\psframebox{\textcolor{Black}{Europe and Asia AM}}}
  \psline{->|}(0.00,12.25)(0.98,12.25)\wrapdnarrow{yellow}{12.25}\rput(.5,12.25){\psframebox{\textcolor{Black}{EU\&Asia AM}}}

  %Radiolocation from OMEGA poster 110-130kHz
  \psline{|<*->|}(7.32,11.05)(9.68,11.05) \rput(8.50,11.05){\psframebox{\textcolor{Black}{Radiolocation}}}

  %Maritime Mobile from OMEGA poster 110-190kHz
  \psline{|<*-}(7.32,11.18)(9.8,11.18) \wrapuparrow{yellow}{11.18}\rput(8.50,11.18){\psframebox{\textcolor{Black}{Maritime Mobile}}}
  \psline{->|}(0.00,11.66)(5.25,11.66)\wrapdnarrow{yellow}{11.66}\rput[l](0,11.66){\psframebox{\textcolor{Black}{Maritime Mobile}}}

  %160-190kHz from http://www.dxing.com/lw.htm
  %replaced by maritime mobile
  %\psline{|<*->|}(2.82,11.65)(5.25,11.65)  \rput(4.53,11.65){\psframebox{\textcolor{Black}{Open US}}}

  %200-430kHz from http://www.dxing.com/lw.htm
  \psline{|<*-}(5.97,11.55)(9.80,11.55)\wrapuparrow{yellow}{11.55} \rput(7.38,11.55){\psframebox{\textcolor{Black}{Navigational Beacons}}}
  \psline{->|}(0.00,12.05)(7.00,12.05)\wrapdnarrow{yellow}{12.05} \rput(4,12.05){\psframebox{\textcolor{Black}{Navigational Beacons}}}

  %235-325kHz
  \psline{|<*-}(8.25,11.65)(9.80,11.65)\wrapuparrow{yellow}{11.65} \rput(9,11.65){\psframebox{\textcolor{Black}{Marine Radio}}}
  \psline{->|}(0.00,12.15)(3.04,12.15)\wrapdnarrow{yellow}{12.15} \rput(1.5,12.15){\psframebox{\textcolor{Black}{Marine Radio}}}

  %430-500kHz from http://www.dxing.com/lw.htm
  \psline{|<*->|}(7.00,12.05)(9.13,12.05) \rput(8.06,12.05){\psframebox{\textcolor{Black}{Morse code}}}

  %500kHz from http://www.dxing.com/lw.htm
  \rput(9.13,12.2){\psframebox[linestyle=none,framesep=1pt,fillcolor=red]{\white SOS}}

  %500-540kHz from http://www.dxing.com/lw.htm
  \psline{|<*-}(9.13,12.05)(9.8,12.05)\wrapuparrow{yellow}{12.05}
  \psline{->|}(0.00,12.55)(0.42,12.55)\wrapdnarrow{yellow}{12.55} \rput(9.56,12.05){\psframebox{\textcolor{Black}{Beacons}}}

  %from http://www.dxing.com/tuning.htm  1700 to 1800 kHz misc beacons
  \psline{|<*->|}(6.83,13.05)(7.64,13.05) \rput(7.24,13.05){\psframebox{\textcolor{Black}{Beacons}}}

  %1800-2000kHz 160m ham band
  %\psline{<->}(7.64,13.05)(9.13,13.05) \rput(8.38,13.05){\psframebox{\textcolor{Black}{160m}}}

  %from http://www.dxing.com/tuning.htm  2000-2300kHz Marine
  \psline{|<*-}(9.13,13.05)(9.80,13.05)\wrapuparrow{yellow}{13.05}\rput(9.49,13.05){\psframebox{\textcolor{Black}{Marine}}}
  \psline{->|}(0.00,13.57)(1.31,13.57)\wrapdnarrow{yellow}{13.57} \rput(0.94,13.57){\psframebox{\textcolor{Black}{Marine}}}

  %from http://www.dxing.com/tuning.htm  2182kHz distress signal
  \rput(0.56,13.65){\psframebox[linestyle=none,framesep=1pt,fillcolor=red]{\white SOS}}

  %from http://www.dxing.com/tuning.htm  2498-2850 kHz More maritime stations
  \psline{|<*->|}(2.47,13.57)(4.34,13.57) \rput(3.40,13.57){\psframebox{\textcolor{Black}{Marine}}}

  %from http://www.dxing.com/tuning.htm  2850-3150 kHz airplane info
  \psline{|<*->|}(4.34,13.57)(5.75,13.57) \rput(5.04,13.57){\psframebox{\textcolor{Black}{Aeronautical}}}

  %from http://www.dxing.com/tuning.htm  3150 to 3200 kHz: This range is allocated to fixed stations, with most communications in RTTY.
  %\psline{|<*->|}(5.75,13.57)(5.97,13.57) \rput(5.83,13.69){\psframebox{\textcolor{Black}{TTY}}}
  \psframe[linestyle=solid,linecolor=yellow,fillstyle=hlines,hatchangle=45,hatchcolor=yellow](5.75,13.52)(5.97,13.62)

  %from http://www.dxing.com/tuning.htm 3400-3500 kHz: This range is used for aeronautical
  \psline{|<*->|}(6.83,13.57)(7.24,13.57) \rput(7.04,13.57){\psframebox{\textcolor{Black}{Aero}}}

  %Loran-C submitted by Poul-Henning Kamp
  \psline{|<*->|}(4.48,11.05)(7.32,11.05) \rput(5.90,11.05){\psframebox{\textcolor{Black}{LORAN-C navigation}}}

  %Radionavigation from OMEGA poster 9-14kHz
  \psline{|<*->|}(1.33,9.55)(7.58,9.55) \rput(4.45,9.55){\psframebox{\textcolor{Black}{Radionavigation}}}

  %Maritime Mobile from OMEGA poster 14-19.95kHz
  \psline{|<*-}(7.58,9.55)(9.80,9.55)\wrapuparrow{yellow}{9.55}\rput[l](7.68,9.55){\psframebox{\textcolor{Black}{Maritime Mobile}}}
  \psline{->|}(0.00,10.05)(2.78,10.05)\wrapdnarrow{yellow}{10.05} \rput[l](0,10.05){\psframebox{\textcolor{Black}{Maritime Mobile}}}

  %Maritime Mobile from OMEGA poster 20.05-59kHz
  \psline{|<*-}(2.85,10.05)(9.80,10.05)\wrapuparrow{yellow}{10.05}\rput[l](4,10.05){\psframebox{\textcolor{Black}{Maritime Mobile}}}
  \psline{->|}(0.00,10.55)(8.31,10.55)\wrapdnarrow{yellow}{10.55} \rput[l](0.0,10.55){\psframebox{\textcolor{Black}{Maritime Mobile}}}

  %Maritime Mobile from OMEGA poster 61-70kHz
  \psline{|<*-}(8.79,10.55)(9.80,10.55)\wrapuparrow{yellow}{10.55}\rput[r](9.8,10.55){\psframebox{\textcolor{Black}{Maritime Mobile}}}
  \psline{->|}(0.00,11.05)(4.48,11.05)\wrapdnarrow{yellow}{11.05} \rput(1,11.05){\psframebox{\textcolor{Black}{Maritime Mobile}}}



 %from http://www.dxing.com/tuning.htm 4000 to 4063 kHz:  mainly used by military forces
  %not enough room to show this one
  %\psline{|<*->|}(9.13,13.57)(9.35,13.57) \rput(9.24,13.57){\psframebox{\textcolor{Black}{MIL}}}

  %4.00-4.438MHz from Radioshack PatrolMan SW-60 shortwave radio guide
  \psline{|<*-}(9.13,13.6)(9.80,13.6)\wrapuparrow{yellow}{13.6} \rput(9.46,13.6){\psframebox{\textcolor{Black}{Seefunk}}}
  \psline{->|}(0.00,14.1)(0.80,14.1)\wrapdnarrow{yellow}{14.1} \rput(.4,14.1){\psframebox{\textcolor{Black}{Seefunk}}}

  %from http://www.dxing.com/tuning.htm 4438 to 4650 kHz: This range is mainly used for fixed and mobile
  %\psline{|<*->|}(0.80,14.05)(1.46,14.05)\rput(1.13,14.15){\psframebox{\textcolor{Black}{Fixed \&}}}\rput(1.13,14.05){\psframebox{\textcolor{Black}{Mobile}}}

  %from http://www.dxing.com/tuning.htm 5005 to 5450 kHz: Misc.
  %\psline{|<*->|}(2.50,14.05)(3.70,14.05)\rput(3.10,14.05){\psframebox{\textcolor{Black}{Misc.}}}
  \psframe[linestyle=solid,linecolor=yellow,fillstyle=hlines,hatchangle=45,hatchcolor=yellow](0.80,14.05)(3.70,14.15)

  %from http://www.dxing.com/tuning.htm 5450 to 5730 kHz: This is another band for aeronautical
  \psline{|<*->|}(3.70,14.1)(4.41,14.1)\rput(4.06,14.1){\psframebox{\textcolor{Black}{Aero}}}

  %from http://www.dxing.com/tuning.htm 5730 to 5950 kHz: Misc.
  %\psline{|<*->|}(4.41,14.2)(4.94,14.2)\rput(4.68,14.2){\psframebox{\textcolor{Black}{Misc.}}}
  \psframe[linestyle=solid,linecolor=yellow,fillstyle=hlines,hatchangle=45,hatchcolor=yellow](4.41,14.15)(4.94,14.25)

  %from http://www.dxing.com/tuning.htm 6200 to 6525 kHz: maritime
  \psline{|<*->|}(5.53,14.2)(6.25,14.2)\rput(5.89,14.2){\psframebox{\textcolor{Black}{Seefunk}}}

  %from http://www.dxing.com/tuning.htm 6525 to 6765 kHz: aeronautical
  \psline{|<*->|}(6.25,14.2)(6.76,14.2)\rput(6.50,14.2){\psframebox{\textcolor{Black}{Aero}}}

  %from http://www.dxing.com/tuning.htm 6765 to 7000 kHz: Misc
  %\psline{|<*->|}(6.76,14.2)(7.24,14.2)\rput(7.00,14.2){\psframebox{\textcolor{Black}{Misc.}}}
  %from http://www.dxing.com/tuning.htm 7300 to 8195 kHz: Misc
  %\psline{|<*->|}(7.83,14.2)(9.47,14.2)\rput(8.65,14.2){\psframebox{\textcolor{Black}{Misc.}}}
  \psframe[linestyle=solid,linecolor=yellow,fillstyle=hlines,hatchangle=45,hatchcolor=yellow](6.76,14.15)(9.47,14.25)

  %from http://www.dxing.com/tuning.htm 8195 to 8815 kHz: maritime
  \psline{|<*-}(9.47,14.2)(9.80,14.2)\wrapuparrow{yellow}{14.2}
  \psline{->|}(0.00,14.75)(0.70,14.75)\wrapdnarrow{yellow}{14.75} \rput(.35,14.75){\psframebox{\textcolor{Black}{Seefunk}}}

  %from http://www.dxing.com/tuning.htm 8815-9040 kHz: This is another aeronautical
  \psline{|<*->|}(0.70,14.75)(1.06,14.75)\rput(0.88,14.75){\psframebox{\textcolor{Black}{Aero}}}

  %from http://www.dxing.com/tuning.htm 9040-9500kHz international broadcasters.
  %from http://www.dxing.com/tuning.htm 9500 to 9900 kHz: This is the 31-meter international broadcasting
  %from http://www.dxing.com/tuning.htm 9900 to 9995 kHz: International broadcasters, FSK modes
  %from http://www.dxing.com/tuning.htm 10005 to 10100 kHz: This range is used for aeronautical
  %too small to fit in
  \psline{|<*->|}(1.06,14.75)(2.62,14.75)\rput(1.77,14.75){\psframebox{\textcolor{Black}{International}}}

  %from http://www.dxing.com/tuning.htm 10150 to 11175 kHz: International and relays
  \psline{|<*->|}(2.69,14.65)(4.05,14.65)\rput(3.37,14.65){\psframebox{\textcolor{Black}{Intnl. and relays}}}

  %from http://www.dxing.com/tuning.htm 11175 to 11400 kHz: This range is used for aeronautical
  \psline{|<*->|}(4.05,14.65)(4.34,14.65)\rput(4.20,14.75){\psframebox{\textcolor{Black}{Flugfunk}}}

  %from http://www.dxing.com/tuning.htm 11400 to 11650 kHz International
  %\psline{|<*->|}(4.34,14.65)(4.64,14.65)\rput(4.49,14.75){\psframebox{\textcolor{Black}{Intnl}}}
  %from http://www.dxing.com/tuning.htm 11975 to 12330 kHz: This band is primarily used by fixed
  %\psline{|<*->|}(5.03,14.7)(5.45,14.7)\rput(5.24,14.7){\psframebox{\textcolor{Black}{Fixed}}}
  \psframe[linestyle=solid,linecolor=yellow,fillstyle=hlines,hatchangle=45,hatchcolor=yellow](4.34,14.65)(5.45,14.75)

  %from http://www.dxing.com/tuning.htm 12330 to 13200 kHz: This is a busy maritime
  \psline{|<*->|}(5.45,14.7)(6.41,14.7)\rput(5.93,14.7){\psframebox{\textcolor{Black}{Seefunk}}}

  %Too small and crammed to show
  %from http://www.dxing.com/tuning.htm 13200 to 13360 kHz: Aeronautical
  %\psline{|<*->|}(6.41,14.7)(6.58,14.7)\rput(6.49,14.7){\psframebox{\textcolor{Black}{Aero}}}
  %from http://www.dxing.com/tuning.htm 13360 to 13600 kHz: This range is used by fixed stations,
  %\psline{|<*->|}(6.58,14.7)(6.83,14.7)\rput(6.71,14.7){\psframebox{\textcolor{Black}{Fixed}}}
  %from http://www.dxing.com/tuning.htm 13800 to 14000 kHz: This is used by fixed
  %\psline{|<*->|}(7.04,14.7)(7.24,14.7)\rput(7.14,14.7){\psframebox{\textcolor{Black}{Fixed}}}
  %from http://www.dxing.com/tuning.htm 14350 to 14990 kHz: This segment is used by fixed stations,
  %\psline{|<*->|}(7.59,14.7)(8.21,14.7)\rput(7.90,14.7){\psframebox{\textcolor{Black}{Fixed}}}
  %from http://www.dxing.com/tuning.htm 15010 to 15100 kHz: This range is for aeronautical
  %\psline{|<*->|}(8.23,14.7)(8.31,14.7)\rput(8.27,14.7){\psframebox{\textcolor{Black}{Aero}}}
  %from http://www.dxing.com/tuning.htm 15600 to 16460 kHz: This band is used by fixed
  %\psline{|<*->|}(8.77,14.7)(9.53,14.7)\rput(9.15,14.7){\psframebox{\textcolor{Black}{Fixed}}}
  \psframe[linestyle=solid,linecolor=yellow,fillstyle=hlines,hatchangle=45,hatchcolor=yellow](6.41,14.65)(9.53,14.75)

  %from http://www.dxing.com/tuning.htm 16460 to 17360 kHz: This range is shared between maritime and fixed
  \psline{|<*-}(9.53,14.7)(9.80,14.7)\wrapuparrow{yellow}{14.7}
  \psline{->|}(0.00,15.2)(0.48,15.2)\wrapdnarrow{yellow}{15.2}\rput(0.24,15.2){\psframebox{\textcolor{Black}{Seefunk}}}

  %from http://www.dxing.com/tuning.htm 17360 to 17550 kHz: The range is shared by aeronautical and fixed
  %\psline{|<*->|}(0.48,15.2)(0.64,15.2)\rput(0.56,15.2){\psframebox{\textcolor{Black}{Aero}}}
  %from http://www.dxing.com/tuning.htm 17900 to 18030 kHz: This band is used for aeronautical
  %\psline{|<*->|}(0.92,15.2)(1.02,15.2)\rput(0.97,15.2){\psframebox{\textcolor{Black}{Aero}}}
  %from http://www.dxing.com/tuning.htm 18030 to 18068 kHz: This range is used by fixed stations
  %\psline{|<*->|}(1.02,15.2)(1.05,15.2)\rput(1.03,15.2){\psframebox{\textcolor{Black}{Fixed}}}
  %from http://www.dxing.com/tuning.htm 18168 to 19990 kHz: Misc.
  %\psline{|<*->|}(1.13,15.2)(2.48,15.2)\rput(1.80,15.2){\psframebox{\textcolor{Black}{Misc}}}
  \psframe[linestyle=solid,linecolor=yellow,fillstyle=hlines,hatchangle=45,hatchcolor=yellow](0.48,15.15)(2.48,15.25)

  %from http://www.dxing.com/tuning.htm 20010 to 21000 kHz: This range is mainly used by fixed stations and a few aeronautical stations
  \psline{|<*->|}(2.49,15.2)(3.17,15.2)\rput(2.83,15.2){\psframebox{\textcolor{Black}{Aero}}}

  %from http://www.dxing.com/tuning.htm 21850 to 22000 kHz: This band is shared by fixed and aeronautical stations
  %\psline{|<*->|}(3.74,15.2)(3.83,15.2)\rput(3.78,15.2){\psframebox{\textcolor{Black}{Aero}}}
  \psframe[linestyle=solid,linecolor=yellow,fillstyle=hlines,hatchangle=45,hatchcolor=yellow](3.74,15.15)(3.83,15.25)

  %from http://www.dxing.com/tuning.htm 22000 to 22855 kHz: This range is reserved for maritime
  \psline{|<*->|}(3.83,15.2)(4.37,15.2)\rput(4.10,15.2){\psframebox{\textcolor{Black}{Marine}}}

  %from http://www.dxing.com/tuning.htm 22855 to 23200 kHz: This band is used by fixed
  %\psline{|<*->|}(4.37,15.2)(4.58,15.2)\rput(4.48,15.2){\psframebox{\textcolor{Black}{Fixed}}}
  %from http://www.dxing.com/tuning.htm 23200 to 23350 kHz: Aeronautical
  %\psline{|<*->|}(4.58,15.2)(4.67,15.2)\rput(4.63,15.2){\psframebox{\textcolor{Black}{Aero}}}
  %from http://www.dxing.com/tuning.htm 23350 to 24890 kHz: This segment is used by fixed stations in FSK and digital modes.
  %\psline{|<*->|}(4.67,15.2)(5.58,15.2)\rput(5.13,15.2){\psframebox{\textcolor{Black}{Fixed}}}
  %from http://www.dxing.com/tuning.htm 25010 to 25550 kHz: This band is used by fixed, mobile, and maritime stations
  %\psline{|<*->|}(5.64,15.2)(5.95,15.2)\rput(5.80,15.2){\psframebox{\textcolor{Black}{Mixed}}}
  %******************************************************from http://www.dxing.com/tuning.htm 25550 to 25670 kHz: This region is reserved for radio astronomy and is usually free of stations.
  %too small, no room
  %\psline{|<*->|}(5.95,15.2)(6.01,15.2)\rput(5.98,15.2){\psframebox{\textcolor{Black}{Astro}}}
  %from http://www.dxing.com/tuning.htm 26100 to 28000 kHz: This band is used by fixed, mobile, and maritime stations
  %\psline{|<*->|}(6.25,15.2)(7.24,15.2)\rput(6.74,15.2){\psframebox{\textcolor{Black}{Mixed}}}
  %from http://www.dxing.com/tuning.htm 29700 to 30000 kHz: This range is used by low powered fixed and mobile stations
  %\psline{|<*->|}(8.07,15.2)(8.22,15.2)\rput(8.15,15.2){\psframebox{\textcolor{Black}{Mixed}}}
  \psframe[linestyle=solid,linecolor=yellow,fillstyle=hlines,hatchangle=45,hatchcolor=yellow](4.37,15.15)(8.22,15.25)

  %from http://www.dxing.com/tuning.htm 72 to 76 MHz: This range is used for remote control signals for model airplanes and garage door openers, wireless microphones (including those used by law enforcement agencies), and two-way communications inside factories, warehouses, and other industrial facilities. Most channels are spaced at 20 kHz intervals.
  \psframe[framearc=0.25](0.99,16.0)(1.76,16.1)\rput(1.38,16.05){\textcolor{Black}{\tiny Remote Ctrl}}

}

% TODO: Textfarbe explizit wieder auf Schwarz zurücksetzen?

%%%%%%%%%%%%%%%%%%%%%%%%%%%%%%%%%%%%%%%%%%%%%%%%%%%%%%%%%%%%%%%%%%%%%%%%%%%%%
% Broadcast radio frequency bands
%%%%%%%%%%%%%%%%%%%%%%%%%%%%%%%%%%%%%%%%%%%%%%%%%%%%%%%%%%%%%%%%%%%%%%%%%%%%%

% Longwave broadcasting band (ITU Region 1)
% Langwellen-Radio
%
% Dieser Frequenzbereich wird nur aus historischen und Orientierungsgründen noch angezeigt.
% Alle praktisch relevanten Langwellensender sind inzwischen abgeschaltet.
%

{

	\psset{linewidth=1pt, linestyle=solid}

	% Langwellenbereich von 148,5 bis 283,5 kHz

	\psframe[fillstyle=solid, fillcolor=BroadcastColour,linewidth=0pt,linestyle=none](1.746,11.55)(9.787,11.75)
	\wrapuparrow{BroadcastColour}{12}
	\psframe[fillstyle=solid, fillcolor=BroadcastColour,linewidth=0pt,linestyle=none](0,12.05)(1.107,12.25)
	\wrapdnarrow{BroadcastColour}{12}

	\rput(5.77,11.65){Langwellenradio}

	\psline(2.187,11.550)(2.187,11.575)% 	point=153000.00
	\rput(2.187,11.65){153}
	\psline(2.995,11.550)(2.995,11.575)% 	point=162000.00
	\rput(2.995,11.650){162}
	\psline(3.760,11.550)(3.760,11.575)% 	point=171000.00
	\rput(3.760,11.650){171}
	\psline(4.485,11.550)(4.485,11.575)% 	point=180000.00
	\rput(4.485,11.650){180}
	\psline(5.175,11.550)(5.175,11.575)% 	point=189000.00
	\rput(5.175,11.650){189}
	\psline(5.832,11.550)(5.832,11.575)% 	point=198000.00
	\rput(5.832,11.650){198}
	\psline(6.461,11.550)(6.461,11.575)% 	point=207000.00
	\rput(6.461,11.650){207}
	\psline(7.063,11.550)(7.063,11.575)% 	point=216000.00
	\rput(7.063,11.650){216}
	\psline(7.640,11.550)(7.640,11.575)% 	point=225000.00
	\rput(7.640,11.650){225}
	\psline(8.194,11.550)(8.194,11.575)% 	point=234000.00
	\rput(8.194,11.650){234}
	\psline(8.728,11.550)(8.728,11.575)% 	point=243000.00
	\rput(8.728,11.650){243}
	\psline(9.242,11.550)(9.242,11.575)% 	point=252000.00
	\rput(9.242,11.650){252}
	\psline(9.738,11.550)(9.738,11.575)% 	point=261000.00
	\rput(9.730,11.650){261}

	\psline(0.475,12.050)(0.475,12.075)% 	point=271100.00
	\rput(0.475,12.150){271}
	\psline(0.937,12.050)(0.937,12.075)% 	point=280100.00
	\rput(0.937,12.150){280}
}

% AM broadcast radio on medium wave
%AM Radio information
{

\psset{linewidth=1pt, linestyle=solid}

  % AM radio 540-1600kHz (two rows)
  %from http://www.dxing.com/tuning.htm
  \definecolor{Fill}{rgb}{0.4,0.68,0.86}
  \psframe[fillstyle=solid, fillcolor=Fill,linewidth=0pt,linestyle=none](0.42,12.55)(9.8,12.75)\wrapuparrow{Fill}{12.65}
  \psframe[fillstyle=solid, fillcolor=Fill,linewidth=0pt,linestyle=none](0.0,13.05)(5.97,13.25)\wrapdnarrow{Fill}{13.15}
  %\rput(0.15,12.65){530kHz}
  %\rput(6.91,13.15){1710kHz}

  % Extended band up to 1610-1710kHz
  \psframe[fillstyle=crosshatch, linewidth=0pt,linestyle=none, hatchwidth=2pt, hatchsep=1.5pt,hatchcolor=Fill](6.06,13.05)(6.83,13.25)

  %\rput(4.9,12.7){AM Radio}
  \rput(4.9,12.75){\psframebox[framesep=2pt,framearc=0.2,fillstyle=solid, fillcolor=Fill,linewidth=0pt,linestyle=none]{Mittelwellenradio}}

%draw baselines on each row for scale
\psline{|<*-}(0.42,12.55)(9.80,12.55)
\psline{->|*}(0.00,13.05)(5.97,13.05)

  \rput(0.42,12.65){\psframebox[framesep=1pt,framearc=0,fillstyle=solid, fillcolor=Fill,linewidth=0pt,linestyle=none]{540}}
  \rput(1.91,12.65){600}
  \rput(4.09,12.65){700}
  \rput(5.97,12.65){800}
  \rput(7.64,12.65){900}
  \rput(9.13,12.65){1000}
  \rput(0.68,13.15){1100}
  \rput(1.91,13.15){1200}
  \rput(3.04,13.15){1300}
  \rput(4.09,13.15){1400}
  \rput(5.06,13.15){1500}
  \rput(5.97,13.15){\psframebox[framesep=1pt,framearc=0,fillstyle=solid, fillcolor=Fill,linewidth=0pt,linestyle=none]{1600}}

%I noticed that my AM car radio starts at 530kHz and goes up by 10kHz to 1710kHz
%generated by scale.c with settings of 530e3 1710e3 10e3

%Note: 1600-1710 is the `recently' expanded band


%\psline(0.15,12.55)(0.15,12.57)% 	point=530000.00 	position=19.02 	ypos=19
\psline(0.42,12.55)(0.42,12.57)% 	point=540000.00 	position=19.04 	ypos=19
\psline(0.68,12.55)(0.68,12.57)% 	point=550000.00 	position=19.07 	ypos=19
\psline(0.93,12.55)(0.93,12.57)% 	point=560000.00 	position=19.10 	ypos=19
\psline(1.18,12.55)(1.18,12.57)% 	point=570000.00 	position=19.12 	ypos=19
\psline(1.43,12.55)(1.43,12.57)% 	point=580000.00 	position=19.15 	ypos=19
\psline(1.67,12.55)(1.67,12.57)% 	point=590000.00 	position=19.17 	ypos=19
\psline(1.91,12.55)(1.91,12.60)% 	point=600000.00 	position=19.19 	ypos=19
\psline(2.14,12.55)(2.14,12.57)% 	point=610000.00 	position=19.22 	ypos=19
\psline(2.37,12.55)(2.37,12.57)% 	point=620000.00 	position=19.24 	ypos=19
\psline(2.60,12.55)(2.60,12.57)% 	point=630000.00 	position=19.26 	ypos=19
\psline(2.82,12.55)(2.82,12.57)% 	point=640000.00 	position=19.29 	ypos=19
\psline(3.04,12.55)(3.04,12.57)% 	point=650000.00 	position=19.31 	ypos=19
\psline(3.25,12.55)(3.25,12.57)% 	point=660000.00 	position=19.33 	ypos=19
\psline(3.47,12.55)(3.47,12.57)% 	point=670000.00 	position=19.35 	ypos=19
\psline(3.68,12.55)(3.68,12.57)% 	point=680000.00 	position=19.38 	ypos=19
\psline(3.88,12.55)(3.88,12.57)% 	point=690000.00 	position=19.40 	ypos=19
\psline(4.09,12.55)(4.09,12.60)% 	point=700000.00 	position=19.42 	ypos=19
\psline(4.29,12.55)(4.29,12.57)% 	point=710000.00 	position=19.44 	ypos=19
\psline(4.48,12.55)(4.48,12.57)% 	point=720000.00 	position=19.46 	ypos=19
\psline(4.68,12.55)(4.68,12.57)% 	point=730000.00 	position=19.48 	ypos=19
\psline(4.87,12.55)(4.87,12.57)% 	point=740000.00 	position=19.50 	ypos=19
\psline(5.06,12.55)(5.06,12.57)% 	point=750000.00 	position=19.52 	ypos=19
\psline(5.25,12.55)(5.25,12.57)% 	point=760000.00 	position=19.54 	ypos=19
\psline(5.43,12.55)(5.43,12.57)% 	point=770000.00 	position=19.55 	ypos=19
\psline(5.62,12.55)(5.62,12.57)% 	point=780000.00 	position=19.57 	ypos=19
\psline(5.80,12.55)(5.80,12.57)% 	point=790000.00 	position=19.59 	ypos=19
\psline(5.97,12.55)(5.97,12.60)% 	point=800000.00 	position=19.61 	ypos=19
\psline(6.15,12.55)(6.15,12.57)% 	point=810000.00 	position=19.63 	ypos=19
\psline(6.32,12.55)(6.32,12.57)% 	point=820000.00 	position=19.65 	ypos=19
\psline(6.49,12.55)(6.49,12.57)% 	point=830000.00 	position=19.66 	ypos=19
\psline(6.66,12.55)(6.66,12.57)% 	point=840000.00 	position=19.68 	ypos=19
\psline(6.83,12.55)(6.83,12.57)% 	point=850000.00 	position=19.70 	ypos=19
\psline(7.00,12.55)(7.00,12.57)% 	point=860000.00 	position=19.71 	ypos=19
\psline(7.16,12.55)(7.16,12.57)% 	point=870000.00 	position=19.73 	ypos=19
\psline(7.32,12.55)(7.32,12.57)% 	point=880000.00 	position=19.75 	ypos=19
\psline(7.48,12.55)(7.48,12.57)% 	point=890000.00 	position=19.76 	ypos=19
\psline(7.64,12.55)(7.64,12.60)% 	point=900000.00 	position=19.78 	ypos=19
\psline(7.80,12.55)(7.80,12.57)% 	point=910000.00 	position=19.80 	ypos=19
\psline(7.95,12.55)(7.95,12.57)% 	point=920000.00 	position=19.81 	ypos=19
\psline(8.10,12.55)(8.10,12.57)% 	point=930000.00 	position=19.83 	ypos=19
\psline(8.25,12.55)(8.25,12.57)% 	point=940000.00 	position=19.84 	ypos=19
\psline(8.40,12.55)(8.40,12.57)% 	point=950000.00 	position=19.86 	ypos=19
\psline(8.55,12.55)(8.55,12.57)% 	point=960000.00 	position=19.87 	ypos=19
\psline(8.70,12.55)(8.70,12.57)% 	point=970000.00 	position=19.89 	ypos=19
\psline(8.84,12.55)(8.84,12.57)% 	point=980000.00 	position=19.90 	ypos=19
\psline(8.99,12.55)(8.99,12.57)% 	point=990000.00 	position=19.92 	ypos=19
\psline(9.13,12.55)(9.13,12.60)% 	point=1000000.00 	position=19.93 	ypos=19
\psline(9.27,12.55)(9.27,12.57)% 	point=1010000.00 	position=19.95 	ypos=19
\psline(9.41,12.55)(9.41,12.57)% 	point=1020000.00 	position=19.96 	ypos=19
\psline(9.55,12.55)(9.55,12.57)% 	point=1030000.00 	position=19.97 	ypos=19
\psline(9.68,12.55)(9.68,12.57)% 	point=1040000.00 	position=19.99 	ypos=19
\psline(0.02,13.05)(0.02,13.07)% 	point=1050000.00 	position=20.00 	ypos=20
\psline(0.15,13.05)(0.15,13.07)% 	point=1060000.00 	position=20.02 	ypos=20
\psline(0.29,13.05)(0.29,13.07)% 	point=1070000.00 	position=20.03 	ypos=20
\psline(0.42,13.05)(0.42,13.07)% 	point=1080000.00 	position=20.04 	ypos=20
\psline(0.55,13.05)(0.55,13.07)% 	point=1090000.00 	position=20.06 	ypos=20
\psline(0.68,13.05)(0.68,13.10)% 	point=1100000.00 	position=20.07 	ypos=20
\psline(0.80,13.05)(0.80,13.07)% 	point=1110000.00 	position=20.08 	ypos=20
\psline(0.93,13.05)(0.93,13.07)% 	point=1120000.00 	position=20.10 	ypos=20
\psline(1.06,13.05)(1.06,13.07)% 	point=1130000.00 	position=20.11 	ypos=20
\psline(1.18,13.05)(1.18,13.07)% 	point=1140000.00 	position=20.12 	ypos=20
\psline(1.31,13.05)(1.31,13.07)% 	point=1150000.00 	position=20.13 	ypos=20
\psline(1.43,13.05)(1.43,13.07)% 	point=1160000.00 	position=20.15 	ypos=20
\psline(1.55,13.05)(1.55,13.07)% 	point=1170000.00 	position=20.16 	ypos=20
\psline(1.67,13.05)(1.67,13.07)% 	point=1180000.00 	position=20.17 	ypos=20
\psline(1.79,13.05)(1.79,13.07)% 	point=1190000.00 	position=20.18 	ypos=20
\psline(1.91,13.05)(1.91,13.10)% 	point=1200000.00 	position=20.19 	ypos=20
\psline(2.02,13.05)(2.02,13.07)% 	point=1210000.00 	position=20.21 	ypos=20
\psline(2.14,13.05)(2.14,13.07)% 	point=1220000.00 	position=20.22 	ypos=20
\psline(2.26,13.05)(2.26,13.07)% 	point=1230000.00 	position=20.23 	ypos=20
\psline(2.37,13.05)(2.37,13.07)% 	point=1240000.00 	position=20.24 	ypos=20
\psline(2.48,13.05)(2.48,13.07)% 	point=1250000.00 	position=20.25 	ypos=20
\psline(2.60,13.05)(2.60,13.07)% 	point=1260000.00 	position=20.26 	ypos=20
\psline(2.71,13.05)(2.71,13.07)% 	point=1270000.00 	position=20.28 	ypos=20
\psline(2.82,13.05)(2.82,13.07)% 	point=1280000.00 	position=20.29 	ypos=20
\psline(2.93,13.05)(2.93,13.07)% 	point=1290000.00 	position=20.30 	ypos=20
\psline(3.04,13.05)(3.04,13.10)% 	point=1300000.00 	position=20.31 	ypos=20
\psline(3.15,13.05)(3.15,13.07)% 	point=1310000.00 	position=20.32 	ypos=20
\psline(3.25,13.05)(3.25,13.07)% 	point=1320000.00 	position=20.33 	ypos=20
\psline(3.36,13.05)(3.36,13.07)% 	point=1330000.00 	position=20.34 	ypos=20
\psline(3.47,13.05)(3.47,13.07)% 	point=1340000.00 	position=20.35 	ypos=20
\psline(3.57,13.05)(3.57,13.07)% 	point=1350000.00 	position=20.36 	ypos=20
\psline(3.68,13.05)(3.68,13.07)% 	point=1360000.00 	position=20.38 	ypos=20
\psline(3.78,13.05)(3.78,13.07)% 	point=1370000.00 	position=20.39 	ypos=20
\psline(3.88,13.05)(3.88,13.07)% 	point=1380000.00 	position=20.40 	ypos=20
\psline(3.99,13.05)(3.99,13.07)% 	point=1390000.00 	position=20.41 	ypos=20
\psline(4.09,13.05)(4.09,13.10)% 	point=1400000.00 	position=20.42 	ypos=20
\psline(4.19,13.05)(4.19,13.07)% 	point=1410000.00 	position=20.43 	ypos=20
\psline(4.29,13.05)(4.29,13.07)% 	point=1420000.00 	position=20.44 	ypos=20
\psline(4.39,13.05)(4.39,13.07)% 	point=1430000.00 	position=20.45 	ypos=20
\psline(4.48,13.05)(4.48,13.07)% 	point=1440000.00 	position=20.46 	ypos=20
\psline(4.58,13.05)(4.58,13.07)% 	point=1450000.00 	position=20.47 	ypos=20
\psline(4.68,13.05)(4.68,13.07)% 	point=1460000.00 	position=20.48 	ypos=20
\psline(4.78,13.05)(4.78,13.07)% 	point=1470000.00 	position=20.49 	ypos=20
\psline(4.87,13.05)(4.87,13.07)% 	point=1480000.00 	position=20.50 	ypos=20
\psline(4.97,13.05)(4.97,13.07)% 	point=1490000.00 	position=20.51 	ypos=20
\psline(5.06,13.05)(5.06,13.10)% 	point=1500000.00 	position=20.52 	ypos=20
\psline(5.16,13.05)(5.16,13.07)% 	point=1510000.00 	position=20.53 	ypos=20
\psline(5.25,13.05)(5.25,13.07)% 	point=1520000.00 	position=20.54 	ypos=20
\psline(5.34,13.05)(5.34,13.07)% 	point=1530000.00 	position=20.55 	ypos=20
\psline(5.43,13.05)(5.43,13.07)% 	point=1540000.00 	position=20.55 	ypos=20
\psline(5.53,13.05)(5.53,13.07)% 	point=1550000.00 	position=20.56 	ypos=20
\psline(5.62,13.05)(5.62,13.07)% 	point=1560000.00 	position=20.57 	ypos=20
\psline(5.71,13.05)(5.71,13.07)% 	point=1570000.00 	position=20.58 	ypos=20
\psline(5.80,13.05)(5.80,13.07)% 	point=1580000.00 	position=20.59 	ypos=20
\psline(5.89,13.05)(5.89,13.07)% 	point=1590000.00 	position=20.60 	ypos=20
\psline(5.97,13.05)(5.97,13.10)% 	point=1600000.00 	position=20.61 	ypos=20
%\psline(6.06,13.05)(6.06,13.07)% 	point=1610000.00 	position=20.62 	ypos=20
%\psline(6.15,13.05)(6.15,13.07)% 	point=1620000.00 	position=20.63 	ypos=20
%\psline(6.24,13.05)(6.24,13.07)% 	point=1630000.00 	position=20.64 	ypos=20
%\psline(6.32,13.05)(6.32,13.07)% 	point=1640000.00 	position=20.65 	ypos=20
%\psline(6.41,13.05)(6.41,13.07)% 	point=1650000.00 	position=20.65 	ypos=20
%\psline(6.49,13.05)(6.49,13.07)% 	point=1660000.00 	position=20.66 	ypos=20
%\psline(6.58,13.05)(6.58,13.07)% 	point=1670000.00 	position=20.67 	ypos=20
%\psline(6.66,13.05)(6.66,13.07)% 	point=1680000.00 	position=20.68 	ypos=20
%\psline(6.75,13.05)(6.75,13.07)% 	point=1690000.00 	position=20.69 	ypos=20
%\psline(6.83,13.05)(6.83,13.10)% 	point=1700000.00 	position=20.70 	ypos=20
%\psline(6.91,13.05)(6.91,13.07)% 	point=1710000.00 	position=20.71 	ypos=20
}

% shortwave radio bands
%%%%%%%%%%%%%%%%%%%%%%%%%%%%%%%%%%%%%%%%%%%%%%%%%%%%%%%%%%%%%%%%%%%%%%%%%%%%%
% Shortwave radio broadcasting bands
%
% This list is focused on ITU region 1. Active bands according to public
% sources and experiences of active listeners
%%%%%%%%%%%%%%%%%%%%%%%%%%%%%%%%%%%%%%%%%%%%%%%%%%%%%%%%%%%%%%%%%%%%%%%%%%%%%

{

	\psset{fillstyle=solid, fillcolor=BroadcastColour}

	\psframe(1.305,13.550)(2.456,13.750) \rput(1.88,13.65){120m}
	\psframe(5.974,13.550)(6.832,13.750) \rput(6.40,13.65){90m}
	\psframe(8.771,13.550)(9.129,13.750) \rput(8.95,13.65){75m}
	\psframe(1.759,14.050)(2.653,14.250) \rput(2.21,14.15){60m}
	\psframe(4.824,14.050)(5.526,14.250) \rput(5.18,14.15){49m}
	\psframe(7.640,14.050)(8.122,14.250) \rput(7.88,14.15){41m}
	\psframe(1.609,14.550)(2.342,14.750) \rput(1.98,14.65){31m}
	\psframe(4.583,14.550)(5.179,14.750) \rput(4.88,14.65){25m}
	\psframe(6.800,14.550)(7.110,14.750) \rput(6.96,14.65){22m}
	\psframe(8.311,14.550)(8.952,14.750) \rput(8.95,14.65){19m}
	\psframe(0.572,15.050)(0.916,15.250) \rput(0.74,15.15){16m}
	\psframe(1.684,15.050)(1.774,15.250) \rput(1.73,15.15){15m}
	\psframe(3.474,15.050)(3.735,15.250) \rput(3.60,15.15){13m}
	\psframe(6.013,15.050)(6.248,15.250) \rput(6.13,15.15){11m}

}

%FM radio
{
\colorlet{UKWColour}{BroadcastColour!90!} % use colour graduations
\psset{linestyle=solid, linewidth=1pt}
\psframe[fillstyle=solid, fillcolor=BroadcastColour,linewidth=0pt,linestyle=none](3.78,16.1)(6.71,16.3)

\rput(5.351,16.3){\psframebox[framesep=2pt,framearc=0.2,fillstyle=solid, fillcolor=UKWColour,linewidth=0pt,linestyle=none]{FM Radio}}

%draw a baseline for the scale
\psline(3.78,16.1)(6.71,16.1)

%Put labels up
\rput(3.784,16.200){\psframebox[framesep=0pt,framearc=0,fillstyle=solid, fillcolor=UKWColour,linestyle=none]{87.7}}
\rput(4.476,16.200){92.1}
\rput(5.077,16.200){96.1}
\rput(5.653,16.200){100.1}
\rput(6.207,16.200){104.1}
\rput(6.714,16.200){\psframebox[framesep=0pt,framearc=0,fillstyle=solid, fillcolor=UKWColour,linestyle=none]{107.9}}

%I noticed that my FM car radio starts at 87.7MHz and goes up by 0.2MHz to 107.9MHz
%
%The following lines of code are generated by scale.c with entered settings of 87.7e6 107.9e6 0.2e6
%They have been shifted vertically with a search and replace on the Y co-ordinates

\psline(3.784,16.100)(3.784,16.15)% 	point=87700000.00
\psline(3.816,16.100)(3.816,16.125)% 	point=87900000.00
\psline(3.848,16.100)(3.848,16.125)% 	point=88100000.00
\psline(3.880,16.100)(3.880,16.125)% 	point=88300000.00
\psline(3.912,16.100)(3.912,16.125)% 	point=88500000.00
\psline(3.944,16.100)(3.944,16.125)% 	point=88700000.00
\psline(3.976,16.100)(3.976,16.125)% 	point=88900000.00
\psline(4.007,16.100)(4.007,16.125)% 	point=89100000.00
\psline(4.039,16.100)(4.039,16.125)% 	point=89300000.00
\psline(4.071,16.100)(4.071,16.125)% 	point=89500000.00
\psline(4.102,16.100)(4.102,16.125)% 	point=89700000.00
\psline(4.134,16.100)(4.134,16.125)% 	point=89900000.00
\psline(4.165,16.100)(4.165,16.125)% 	point=90100000.00
\psline(4.197,16.100)(4.197,16.125)% 	point=90300000.00
\psline(4.228,16.100)(4.228,16.125)% 	point=90500000.00
\psline(4.259,16.100)(4.259,16.125)% 	point=90700000.00
\psline(4.290,16.100)(4.290,16.125)% 	point=90900000.00
\psline(4.321,16.100)(4.321,16.125)% 	point=91100000.00
\psline(4.352,16.100)(4.352,16.125)% 	point=91300000.00
\psline(4.383,16.100)(4.383,16.125)% 	point=91500000.00
\psline(4.414,16.100)(4.414,16.125)% 	point=91700000.00
\psline(4.445,16.100)(4.445,16.125)% 	point=91900000.00
\psline(4.476,16.100)(4.476,16.15)% 	point=92100000.00
\psline(4.506,16.100)(4.506,16.125)% 	point=92300000.00
\psline(4.537,16.100)(4.537,16.125)% 	point=92500000.00
\psline(4.567,16.100)(4.567,16.125)% 	point=92700000.00
\psline(4.598,16.100)(4.598,16.125)% 	point=92900000.00
\psline(4.628,16.100)(4.628,16.125)% 	point=93100000.00
\psline(4.659,16.100)(4.659,16.125)% 	point=93300000.00
\psline(4.689,16.100)(4.689,16.125)% 	point=93500000.00
\psline(4.719,16.100)(4.719,16.125)% 	point=93700000.00
\psline(4.749,16.100)(4.749,16.125)% 	point=93900000.00
\psline(4.779,16.100)(4.779,16.125)% 	point=94100000.00
\psline(4.809,16.100)(4.809,16.125)% 	point=94300000.00
\psline(4.839,16.100)(4.839,16.125)% 	point=94500000.00
\psline(4.869,16.100)(4.869,16.125)% 	point=94700000.00
\psline(4.899,16.100)(4.899,16.125)% 	point=94900000.00
\psline(4.929,16.100)(4.929,16.125)% 	point=95100000.00
\psline(4.959,16.100)(4.959,16.125)% 	point=95300000.00
\psline(4.988,16.100)(4.988,16.125)% 	point=95500000.00
\psline(5.018,16.100)(5.018,16.125)% 	point=95700000.00
\psline(5.047,16.100)(5.047,16.125)% 	point=95900000.00
\psline(5.077,16.100)(5.077,16.15)% 	point=96100000.00
\psline(5.106,16.100)(5.106,16.125)% 	point=96300000.00
\psline(5.135,16.100)(5.135,16.125)% 	point=96500000.00
\psline(5.165,16.100)(5.165,16.125)% 	point=96700000.00
\psline(5.194,16.100)(5.194,16.125)% 	point=96900000.00
\psline(5.223,16.100)(5.223,16.125)% 	point=97100000.00
\psline(5.252,16.100)(5.252,16.125)% 	point=97300000.00
\psline(5.281,16.100)(5.281,16.125)% 	point=97500000.00
\psline(5.310,16.100)(5.310,16.125)% 	point=97700000.00
\psline(5.339,16.100)(5.339,16.125)% 	point=97900000.00
\psline(5.368,16.100)(5.368,16.125)% 	point=98100000.00
\psline(5.397,16.100)(5.397,16.125)% 	point=98300000.00
\psline(5.425,16.100)(5.425,16.125)% 	point=98500000.00
\psline(5.454,16.100)(5.454,16.125)% 	point=98700000.00
\psline(5.483,16.100)(5.483,16.125)% 	point=98900000.00
\psline(5.511,16.100)(5.511,16.125)% 	point=99100000.00
\psline(5.540,16.100)(5.540,16.125)% 	point=99300000.00
\psline(5.568,16.100)(5.568,16.125)% 	point=99500000.00
\psline(5.597,16.100)(5.597,16.125)% 	point=99700000.00
\psline(5.625,16.100)(5.625,16.125)% 	point=99900000.00
\psline(5.653,16.100)(5.653,16.15)% 	point=100100000.00
\psline(5.682,16.100)(5.682,16.125)% 	point=100300000.00
\psline(5.710,16.100)(5.710,16.125)% 	point=100500000.00
\psline(5.738,16.100)(5.738,16.125)% 	point=100700000.00
\psline(5.766,16.100)(5.766,16.125)% 	point=100900000.00
\psline(5.794,16.100)(5.794,16.125)% 	point=101100000.00
\psline(5.822,16.100)(5.822,16.125)% 	point=101300000.00
\psline(5.850,16.100)(5.850,16.125)% 	point=101500000.00
\psline(5.877,16.100)(5.877,16.125)% 	point=101700000.00
\psline(5.905,16.100)(5.905,16.125)% 	point=101900000.00
\psline(5.933,16.100)(5.933,16.125)% 	point=102100000.00
\psline(5.961,16.100)(5.961,16.125)% 	point=102300000.00
\psline(5.988,16.100)(5.988,16.125)% 	point=102500000.00
\psline(6.016,16.100)(6.016,16.125)% 	point=102700000.00
\psline(6.043,16.100)(6.043,16.125)% 	point=102900000.00
\psline(6.071,16.100)(6.071,16.125)% 	point=103100000.00
\psline(6.098,16.100)(6.098,16.125)% 	point=103300000.00
\psline(6.126,16.100)(6.126,16.125)% 	point=103500000.00
\psline(6.153,16.100)(6.153,16.125)% 	point=103700000.00
\psline(6.180,16.100)(6.180,16.125)% 	point=103900000.00
\psline(6.207,16.100)(6.207,16.15)% 	point=104100000.00
\psline(6.234,16.100)(6.234,16.125)% 	point=104300000.00
\psline(6.261,16.100)(6.261,16.125)% 	point=104500000.00
\psline(6.289,16.100)(6.289,16.125)% 	point=104700000.00
\psline(6.316,16.100)(6.316,16.125)% 	point=104900000.00
\psline(6.342,16.100)(6.342,16.125)% 	point=105100000.00
\psline(6.369,16.100)(6.369,16.125)% 	point=105300000.00
\psline(6.396,16.100)(6.396,16.125)% 	point=105500000.00
\psline(6.423,16.100)(6.423,16.125)% 	point=105700000.00
\psline(6.450,16.100)(6.450,16.125)% 	point=105900000.00
\psline(6.476,16.100)(6.476,16.125)% 	point=106100000.00
\psline(6.503,16.100)(6.503,16.125)% 	point=106300000.00
\psline(6.530,16.100)(6.530,16.125)% 	point=106500000.00
\psline(6.556,16.100)(6.556,16.125)% 	point=106700000.00
\psline(6.583,16.100)(6.583,16.125)% 	point=106900000.00
\psline(6.609,16.100)(6.609,16.125)% 	point=107100000.00
\psline(6.635,16.100)(6.635,16.125)% 	point=107300000.00
\psline(6.662,16.100)(6.662,16.125)% 	point=107500000.00
\psline(6.688,16.100)(6.688,16.125)% 	point=107700000.00
\psline(6.714,16.100)(6.714,16.15)% 	point=107900000.00


%\rput(3.34,16.20){\psframebox[framearc=0.25,fillstyle=solid, fillcolor=BroadcastColour,linecolor=Black]{85MHz}}
%\rput(8.22,16.20){\psframebox[framearc=0.25,fillstyle=solid, fillcolor=BroadcastColour,linecolor=Black]{120MHz}}

%\rput(3.34,16.2){85MHz}
%\rput(8.22,16.2){120MHz}

}

% Digital Audio Broadcasting -- digital radio (Germany)
% DAB+ Digital Radio
{

\definecolor{DABText}{rgb}{0,0,0}
\definecolor{DABColour}{rgb}{0.4,0.68,0.86} %usual broadcast radio colour

\psset{fillstyle=solid,linestyle=solid,framearc=0.25,fillcolor=DABColour,linecolor=Black,linewidth=0.6pt}

\rput(5.72,16.66){\psframebox[framesep=2pt,framearc=0.2,fillstyle=solid,fillcolor=DABColour,linewidth=0pt,linestyle=none]{DAB+-Radio}}

\tiny	% TODO: Numbers still too wide for the boxes, find narrow font

% VHF-Band III

\psframe(3.683,16.5)(3.807,16.6)		\rput(3.75,16.55){5A}
\psframe(3.822,16.5)(3.944,16.6)		\rput(3.88,16.55){5B}
\psframe(3.958,16.5)(4.080,16.6)		\rput(4.02,16.55){5C}
\psframe(4.094,16.5)(4.215,16.6)		\rput(4.15,16.55){5D}

\psframe(4.241,16.5)(4.360,16.6)		\rput(4.3,16.55){6A}
\psframe(4.374,16.5)(4.492,16.6)		\rput(4.43,16.55){6B}
\psframe(4.506,16.5)(4.623,16.6)		\rput(4.56,16.55){6C}
\psframe(4.636,16.5)(4.752,16.6)		\rput(4.69,16.55){6D}

\psframe(4.776,16.5)(4.891,16.6)		\rput(4.83,16.55){7A}
\psframe(4.904,16.5)(5.018,16.6)		\rput(4.96,16.55){7B}
\psframe(5.031,16.5)(5.144,16.6)		\rput(5.09,16.55){7C}
\psframe(5.157,16.5)(5.269,16.6)		\rput(5.21,16.55){7D}

\psframe(5.293,16.5)(5.404,16.6)		\rput(5.35,16.55){8A}
\psframe(5.417,16.5)(5.527,16.6)		\rput(5.47,16.55){8B}
\psframe(5.539,16.5)(5.648,16.6)		\rput(5.59,16.55){8C}
\psframe(5.661,16.5)(5.769,16.6)		\rput(5.72,16.55){8D}

\psframe(5.791,16.5)(5.898,16.6)		\rput(5.84,16.55){9A}
\psframe(5.910,16.5)(6.016,16.6)		\rput(5.96,16.55){9B}
\psframe(6.028,16.5)(6.134,16.6)		\rput(6.08,16.55){9C}
\psframe(6.146,16.5)(6.250,16.6)		\rput(6.20,16.55){9D}

% Die "N"-Blöcke werden in Europa nicht genutzt
\psframe(6.273,16.5)(6.376,16.6)		\rput(6.32,16.55){10A}
% \psframe(6.284,16.5)(6.387,16.6)		\rput(6.34,16.55){10N}
\psframe(6.388,16.5)(6.491,16.6)		\rput(6.44,16.55){10B}
\psframe(6.502,16.5)(6.604,16.6)		\rput(6.55,16.55){10C}
\psframe(6.616,16.5)(6.717,16.6)		\rput(6.67,16.55){10D}

\psframe(6.738,16.5)(6.838,16.6)		\rput(6.79,16.55){11A}
%\psframe(6.748,16.5)(6.848,16.6)		\rput(6.80,16.55){11N}
\psframe(6.849,16.5)(6.949,16.6)		\rput(6.90,16.55){11B}
\psframe(6.960,16.5)(7.058,16.6)		\rput(7.01,16.55){11C}
\psframe(7.070,16.5)(7.168,16.6)		\rput(7.12,16.55){11D}

\psframe(7.189,16.5)(7.286,16.6)		\rput(7.24,16.55){12A}
% \psframe(7.199,16.5)(7.296,16.6)		\rput(7.25,16.55){12N}
\psframe(7.297,16.5)(7.393,16.6)		\rput(7.35,16.55){12B}
\psframe(7.404,16.5)(7.500,16.6)		\rput(7.45,16.55){12C}	
\psframe(7.511,16.5)(7.605,16.6)		\rput(7.56,16.55){12D}

% Hier schliesst sich noch ein breiter Kanal 13 an, der aber in DACH nicht genutzt wird

% für niedrige Kanalkästchen erster Y-Wert -0,05 zweiter -0,15 Textpos. -0,1

}



%%%%%%%%%%%%%%%%%%%%%%%%%%%%%%%%%%%%%%%%%%%%%%%%%%%%%%%%%%%%%%%%%%%%%%%%%%%%%
% TODO: SORT THOSE
%%%%%%%%%%%%%%%%%%%%%%%%%%%%%%%%%%%%%%%%%%%%%%%%%%%%%%%%%%%%%%%%%%%%%%%%%%%%%


%Power supply transmission frequencies
%Power lines


 % Eisenbahnen in Deutschland
  \psframe[framearc=0, fillstyle=solid, fillcolor=FColor,linewidth=0pt,linestyle=none](0.48,5.0)(1.2,5.3)
  \rput(0.88,5.230){$16\frac{2}{3}$Hz}\rput(0.880,5.08){Bahnstrom}

  % 50Hz Power Lines used in Europe
  \psframe[framearc=0, fillstyle=solid, fillcolor=FColor,linewidth=0pt,linestyle=none](5.9,5.5)(6.72,5.8)
  \rput(6.31,5.73){50Hz Netz}
  \rput(6.02,5.58){
	\psframebox[linestyle=none,fillstyle=none]{
		\psset{linestyle=solid,linecolor=Black,linewidth=1pt,fillstyle=none}
		\psline[linestyle=solid,linecolor=gray](0.2,0)(.64,0)
		\rput(0.05,0){v=}
		\parabola(0.24,0)(0.27,.07)\parabola(0.30,0)(0.33,-.07)
		\parabola(0.36,0)(0.39,.07)\parabola(0.42,0)(0.45,-.07)
		\parabola(0.48,0)(0.51,.07)\parabola(0.54,0)(0.57,-.07)
	}
  }
  \psframe[framearc=0, fillstyle=crosshatch, fillcolor=FColor,linewidth=0pt,linestyle=none, hatchcolor=FColor,hatchsep=1.5pt,hatchwidth=.2pt](5.9,5.8)(6.72,6.0)
  \psframe[framearc=0, fillstyle=solid, fillcolor=FColor,linewidth=0pt,linestyle=none](5.9,6.0)(6.72,6.3)
  \rput(6.31,6.23){100Hz Lampen}
  \rput(6.02,6.08){
	\psframebox[linestyle=none,fillstyle=none]{
		\psset{linestyle=solid,linecolor=Black,linewidth=1pt,fillstyle=none}
		\psline[linestyle=solid,linecolor=gray](0.2,-.07)(.64,-.07)
		\rput(0.05,0){p=}
		\parabola(0.24,0)(.255,.07)\parabola(0.27,0)(.285,-.07)
		\parabola(0.30,0)(.315,.07)\parabola(0.33,0)(.345,-.07)
		\parabola(0.36,0)(.375,.07)\parabola(0.39,0)(.405,-.07)
		\parabola(0.42,0)(.435,.07)\parabola(0.45,0)(.465,-.07)
		\parabola(0.48,0)(.495,.07)\parabola(0.51,0)(.525,-.07)
		\parabola(0.54,0)(.555,.07)\parabola(0.57,0)(.585,-.07)
	}
  }

  %60Hz Power Lines
%   \psframe[framearc=0, fillstyle=solid, fillcolor=FColor,linewidth=0pt,linestyle=none](8.48,5.5)(9.3,5.8)
%   \rput(8.89,5.73){60Hz Power}
%   \rput(8.6,5.58){
% 	\psframebox[linestyle=none,fillstyle=none]{
% 		\psset{linestyle=solid,linecolor=Black,linewidth=1pt,fillstyle=none}
% 		\psline[linestyle=solid,linecolor=gray](0.2,0)(.64,0)
% 		\rput(0.05,0){v=}
% 		\parabola(0.24,0)(0.27,.07)\parabola(0.30,0)(0.33,-.07)
% 		\parabola(0.36,0)(0.39,.07)\parabola(0.42,0)(0.45,-.07)
% 		\parabola(0.48,0)(0.51,.07)\parabola(0.54,0)(0.57,-.07)
% 	}
%   }
%   \psframe[framearc=0, fillstyle=crosshatch, fillcolor=FColor,linewidth=0pt,linestyle=none, hatchcolor=FColor,hatchsep=1.5pt,hatchwidth=.2pt](8.48,5.8)(9.3,6)
%   \psframe[framearc=0, fillstyle=solid, fillcolor=FColor,linewidth=0pt,linestyle=none](8.48,6)(9.3,6.3)
%   \rput(8.89,6.23){120Hz Lights}
%   \rput(8.6,6.08){
% 	\psframebox[linestyle=none,fillstyle=none]{
% 	\psset{linestyle=solid,linecolor=Black,linewidth=1pt,fillstyle=none}
% 	\psline[linestyle=solid,linecolor=gray](0.2,-.07)(.64,-.07)
% 	\rput(0.05,0){p=}
% 	\parabola(0.24,0)(.255,.07)\parabola(0.27,0)(.285,-.07)
% 	\parabola(0.30,0)(.315,.07)\parabola(0.33,0)(.345,-.07)
% 	\parabola(0.36,0)(.375,.07)\parabola(0.39,0)(.405,-.07)
% 	\parabola(0.42,0)(.435,.07)\parabola(0.45,0)(.465,-.07)
% 	\parabola(0.48,0)(.495,.07)\parabola(0.51,0)(.525,-.07)
% 	\parabola(0.54,0)(.555,.07)\parabola(0.57,0)(.585,-.07)
% 	}
%   }
% 
{\psset{fillstyle=none, linecolor=FColor, linestyle=solid, linewidth=.5pt, linearc=1pt}
\psbezier{->}(5.9,5.6)(4.9,5.6)(4.9,6.65)(2.59,6.65)
% \psbezier{->}(8.48,5.6)(7.7,5.6)(7.7,6.65)(5.17,6.65)
} 
% Third harmonics of power lines, suggested by Poul-Henning Kamp
\psframe[framearc=0, fillstyle=solid, fillcolor=FColor,linewidth=0pt,linestyle=none](1.88,6.5)(2.6,6.8)
  \rput(2.24,6.71){150Hz}
  \rput(2.24,6.59){3.Harmon.}
% \psframe[framearc=0, fillstyle=solid, fillcolor=FColor,linewidth=0pt,linestyle=none](4.46,6.5)(5.18,6.8)
%   \rput(4.82,6.71){180Hz}
%   \rput(4.82,6.59){3rd harmonic}


  %Airplane Power
  \psframe[framearc=0, fillstyle=solid, fillcolor=FColor,linewidth=0pt,linestyle=none](5.90,7)(6.72,7.3)
  \rput(6.31,7.23){400Hz}\rput(6.31,7.07){Flugz.Bordnetz}








%some extra radio frequencies, pagers, wireless mics etc.
% \input{tex/itinerant.tex}

%The Schumann resonances
\input{tex/schumann.tex}


%Middle A musical note 440 Hz
\rput(7.66,7.16){\pscirclebox[linestyle=solid, linecolor=white, linewidth=1pt, fillstyle=solid,fillcolor=HumanAudioColor]{\textcolor{white}{A4}}}
%Middle C was 261.63Hz located at \rput(0.31,7.66)


  % Digital Television (DTV) Television (Analog) VHF & UHF (Audio??) 54-72MHz
%  \definecolor{BoxColor}{rgb}{0.5,0.68,0.86}
%  \psframe[fillstyle=solid, fillcolor=BoxColor,linewidth=0pt,linestyle=none](6.73,15.55)(9.8,15.65)
%  \psframe[fillstyle=solid, fillcolor=BoxColor,linewidth=0pt,linestyle=none](0.0,16.05)(0.99,16.15)
%  \rput(6.73,15.60){54MHz}
%  \rput(8,15.60){DTV}
%  \rput(0.99,16.10){72MHz}


  % Digital Television (DTV) Television (Analog) VHF & UHF (Audio??) 76-85 MHz
%  \definecolor{BoxColor}{rgb}{0.6,0.68,0.86}
%  \psframe[fillstyle=solid, fillcolor=BoxColor,linewidth=0pt,linestyle=none](1.76,16.05)(3.34,16.15)
%  \rput(1.76,16.10){76MHz}
%  \rput(2.34,16.10){DTV}

%  \input{tex/tv.tex}

  % Lizenzfreie Funkanwendungen (Situation in Deutschland 2020)

{

	\psset{linewidth=0pt, linestyle=none, fillstyle=solid, fillcolor=LicenseFreeColor}

	% CB-Funk auf 80 Kanälen in D, 26,565-27.405 MHz
	\psframe(6.498,15.05)(6.938,15.25)\rput(6.72,15.15){CB-Funk}

	% Freenet auf 6 Kanälen
	\psframe(1.477,16.55)(1.488,16.75)\rput(1.7,16.65){\color{LicenseFreeColor}Freenet}

	% LPD / SRD
	% Wegen auslaufender Genehmigung und Überlappung mit 70 cm AFU erst mal ausgelassen

	% PMR
	\psframe(7.178,17.05)(7.181,17.25)\rput(7.35,17.15){\color{LicenseFreeColor}PMR}

}

  % Aeronautical Mobile 108-136 MHz
  \definecolor{BoxColor}{rgb}{0.6,0.68,0.86}
  \psframe[fillstyle=solid, fillcolor=BoxColor,linewidth=0pt,linestyle=none](6.73,16.1)(9.80,16.3)\wrapuparrow{BoxColor}{16.2}
  \psframe[fillstyle=solid, fillcolor=BoxColor,linewidth=0pt,linestyle=none](0.00,16.6)(0.19,16.8)\wrapdnarrow{BoxColor}{16.75}
  \rput(9.07,16.2){Aeronautical}

  \rput(8.39,16.2){\psframebox[linestyle=none,framesep=1pt,fillcolor=red,fillstyle=solid]{\white SOS}}



  % Marine mobile 162-174 MHz
  \definecolor{BoxColor}{rgb}{0.6,0.86,0.86}
  \psframe[fillstyle=solid, fillcolor=BoxColor,linewidth=0pt,linestyle=none](2.66,16.6)(3.67,16.7)
  %\rput(2.66,16.70){162MHz}
  \rput(3.17,16.65){Marine Mobile}

  % Digital Television (DTV) Television (Analog) VHF & UHF  174-216 MHz
  %\definecolor{BoxColor}{rgb}{0.6,0.8,0.8}
  %\psframe[fillstyle=solid, fillcolor=BoxColor,linewidth=0pt,linestyle=none](3.67,16.55)(6.73,16.65)
  %\rput(3.67,16.6){174MHz}
  %\rput(5.20,16.6){Digital Television}


\input{tex/astrofilter.tex}

%54-216 MHz  VHF  (***spread this line over many rows!)


  % SuperBand - CATV ch 23-36 (216-300 MHz)
%  \definecolor{BoxColor}{rgb}{0.8,0.68,0.86}
%  \psframe[fillstyle=solid, fillcolor=BoxColor,linewidth=0pt,linestyle=none](6.73,16.55)(9.8,16.65)
%  \psframe[fillstyle=solid, fillcolor=BoxColor,linewidth=0pt,linestyle=none](0.0,17.05)(1.57,17.15)
%  \rput(6.73,16.60){216MHz}
%  \rput(8,16.60){SuperBand - CATV ch 23-36}

  %from http://www.dxing.com/above30.htm 225 to 400 MHz: This very wide band is used for military aviation communications in AM. Most channels are 100 kHz apart.
  \definecolor{BoxColor}{rgb}{0.6,0.68,0.86}
  \psframe[fillstyle=solid, fillcolor=BoxColor,linewidth=0pt,linestyle=none](7.30,16.6)(9.80,16.8)\wrapuparrow{BoxColor}{16.7}
  \psframe[fillstyle=solid, fillcolor=BoxColor,linewidth=0pt,linestyle=none](0.00,17.2)(5.64,17.3)\wrapdnarrow{BoxColor}{17.25}
  \rput(9.07,16.7){Military}
  \rput(8.39,16.7){\psframebox[linestyle=none,framesep=1pt,fillcolor=red,fillstyle=solid]{\white SOS}}
  \rput(2.82,17.25){Military}

%from http://www.dxing.com/above30.htm 400 to 406 MHz: This range is used primarily by government and military stations in FM.
%from http://www.dxing.com/above30.htm 406 to 420 MHz: In the United States, this band is used exclusively by the federal government. All transmissions are in FM, with most channels spaced at 25 kHz intervals.
%   \definecolor{BoxColor}{rgb}{0.9,0.68,0.86}
%   \psframe[fillstyle=solid, fillcolor=BoxColor,linewidth=0pt,linestyle=none](5.64,17.2)(6.33,17.35) \rput(6.15,17.28){Gov}
%   \rput(5.85,17.27){\psframebox[linestyle=none,framesep=1pt,fillcolor=red,fillstyle=solid]{\white SOS}}


  % HyperBand-CATV Ch 37-62 (300-456 MHz)
%  \definecolor{BoxColor}{rgb}{0.9,0.68,0.86}
%  \psframe[fillstyle=solid, fillcolor=BoxColor,linewidth=0pt,linestyle=none](1.57,17.05)(7.49,17.15)
%  %\rput(1.57,17.25){300MHz}
%  \rput(4.16,17.10){HyperBand-CATV Ch 37-6}


  % UHF Band Ch 14-83 (470-890 MHz)
  %\definecolor{BoxColor}{rgb}{0.9,0.68,0.86}
  %\psframe[fillstyle=solid, fillcolor=BoxColor,linewidth=0pt,linestyle=none](7.92,17.05)(9.80,17.15)
  %\psframe[fillstyle=solid, fillcolor=BoxColor,linewidth=0pt,linestyle=none](0.0,17.55)(7.15,17.65)
  %\rput(4.16,17.60){UHF Band Ch 14-83}


  %Hydrogen line at 21cm (1.42758313333 GHz)
  %Idea from Douglas Scott
  %Info from http://www.drao-ofr.hia-iha.nrc-cnrc.gc.ca/outreach/skygazing/1997/sep1197.html
  \blip{4.03,18}{H}

% \input{tex/catv.tex}


  % Cell phone 800MHz (Time division multiple access (TDMA)) not actuially 800MHz, and already covered in the 3G list
%   \rput(5.64,17.5){\psframebox[fillstyle=solid,linecolor=yellow,framesep=0pt,fillcolor=yellow,linewidth=1pt,linestyle=solid]{\textcolor{Black}{CP}}}
  %from http://www.dxing.com/tuning.htm 825-894MHz cell phone including 850MHz GSM phones
  %(Frequency division multiple access (FDMA))
%   \psline[linecolor=yellow,linewidth=1pt,linestyle=solid]{|<*->|}(6.07,17.5)(7.21,17.5)
%   \rput(6.64,17.5){\psframebox[fillstyle=solid,linecolor=yellow,framesep=0pt,fillcolor=yellow,linewidth=1pt,linestyle=solid]{\textcolor{Black}{CP}}}

  % Cell phone 1.8GHz (GSM phone)
%   \rput[b](7.30,18.0){\psframebox[fillstyle=solid,linecolor=yellow,framesep=0pt,fillcolor=yellow,linewidth=1pt,linestyle=solid]{\textcolor{Black}{CP}}}

  % Cell phone 1.9GHz
  %\rput[b](8.07,18.0){\psframebox[fillstyle=solid,linecolor=yellow,framesep=0pt,fillcolor=yellow,linewidth=1pt,linestyle=solid]{\textcolor{Black}{CP}}}

  % Code division multiple access (CDMA) 1850-1990MHz (PCS phones)
  % notes from from http://electronics.howstuffworks.com/cell-phone8.htm
%   \psline[linecolor=yellow,linewidth=1pt,linestyle=solid]{|<*->|}(7.69,18.0)(8.72,18.0)
%   \rput[b](8.21,18.0){\psframebox[fillstyle=solid,linecolor=yellow,framesep=0pt,fillcolor=yellow,linewidth=1pt,linestyle=solid]{\textcolor{Black}{CP}}}


  %from:
  % - Radioshack PatrolMan SW-60 shortwave radio guide
  % - http://www.dxing.com/tuning.htm
  % - Poul-Henning Kamp
  % - OMEGA poster
  \rput(2.81,10.15){\timestandard}% 20.0 kHz  Standard Frequency and Time Signal from OMEGA poster
  \rput(8.55,10.65){\timestandard}% 60.0 kHz  WWVB-USA, MSF-UK  time signal transmissions.
  \rput(1.91,11.15){\timestandard}% 75.0 kHz  Swiss time signal transmission
  \rput(2.37,11.15){\timestandard}% 77.5 kHz  DCF77 - German time signal transmission
  \rput(2.48,13.70){\timestandard}% 2.5MHz
  \rput(6.54,13.70){\timestandard}% 3.33MHz Canadian
  \rput(2.48,14.15){\timestandard}% 5MHz
  \rput(7.90,14.15){\timestandard}% 7.335MHz Canadian
  \rput(2.48,14.65){\timestandard}% 10MHz
  \rput(7.90,14.65){\timestandard}% 14.670MHz Canadian
  \rput(8.22,14.65){\timestandard}% 15MHz
  \rput(2.48,15.15){\timestandard}% 20MHz Reception here is usually possible only in daytime.
  \rput(5.64,15.15){\timestandard}% 25MHz This range is for standard time and frequency stations, although none are currently operating here.




  %water absorption band at 2.4GHz water (that's why microwave ovens use that band). Thanks to Poul-Henning Kamp
% Later proven wrong by Anders J. Johansson (20060219)
%   \rput(0.57,18.55){
%   \psclip{\psframe[fillstyle=none,linestyle=none](0,0)(2.4,.3)}
%     \pscurve[fillstyle=crosshatch,hatchcolor=Black,hatchangle=180,hatchwidth=0.5pt,hatchsep=1pt,fillcolor=blue,linecolor=Black,linestyle=solid]
% 	(0,.2)(.5,.15)(1,.02)(1.5,.15)(2,0.2)
%     \rput(.5,.2){\psframebox[framesep=1pt,framearc=0.2,fillstyle=solid, fillcolor=Black,linewidth=0pt,linestyle=none]{\textcolor{white}{Water absorption}}}
%
%   \endpsclip}


  % Marker for ISM band, TODO: Move to
  \psframe[fillstyle=solid, fillcolor=PineGreen](1.572,18.55)(2.055,18.75)


  %Wireless LAN   Frequency Range; 2.4GHz to 2.484GHz , 5.15GHz to 5.875GHz  (From D-Link Air expert card, good for America and Europe)
  %2.4GHz to 2.5GHz
  \psframe[fillstyle=solid, fillcolor=gray](1.572,18.55)(2.055,18.75)\rput(1.81,18.65){\textcolor{yellow}{W-LAN}}
  %5.15GHz to 5.875GHz
  \psframe[fillstyle=solid, fillcolor=gray](2.57,19.15)(4.43,19.25)\rput(3.50,19.20){\textcolor{yellow}{W-LAN}}

{%C-Band satellite channels
\newlength{\nCBANDOddYOffset} \setlength{\nCBANDOddYOffset}{18.60in}
\newlength{\nCBANDEvenYOffset} \setlength{\nCBANDEvenYOffset}{\nCBANDOddYOffset+0.08in}
\tiny
  \psset{framesep=1pt,framearc=.25,fillcolor=green,fillstyle=solid}
  \rput(7.77,\nCBANDOddYOffset){\psframebox{01}}	%3720MHz
  \rput(7.84,\nCBANDEvenYOffset){\psframebox{02}}	%3740
  \rput(7.92,\nCBANDOddYOffset){\psframebox{03}}	%3760
  \rput(7.99,\nCBANDEvenYOffset){\psframebox{04}}	%3780
  \rput(8.07,\nCBANDOddYOffset){\psframebox{05}}	%3800
  \rput(8.14,\nCBANDEvenYOffset){\psframebox{06}}	%3820
  \rput(8.22,\nCBANDOddYOffset){\psframebox{07}}	%3840
  \rput(8.29,\nCBANDEvenYOffset){\psframebox{08}}	%3860
  \rput(8.36,\nCBANDOddYOffset){\psframebox{09}}	%3880
  \rput(8.44,\nCBANDEvenYOffset){\psframebox{10}}	%3900
  \rput(8.51,\nCBANDOddYOffset){\psframebox{11}}	%3920
  \rput(8.58,\nCBANDEvenYOffset){\psframebox{12}}	%3940
  \rput(8.65,\nCBANDOddYOffset){\psframebox{13}}	%3960
  \rput(8.72,\nCBANDEvenYOffset){\psframebox{14}}	%3980
  \rput(8.79,\nCBANDOddYOffset){\psframebox{15}}	%4000
  \rput(8.86,\nCBANDEvenYOffset){\psframebox{16}}	%4020
  \rput(8.93,\nCBANDOddYOffset){\psframebox{17}}	%4040
  \rput(9.00,\nCBANDEvenYOffset){\psframebox{18}}	%4060
  \rput(9.07,\nCBANDOddYOffset){\psframebox{19}}	%4080
  \rput(9.14,\nCBANDEvenYOffset){\psframebox{20}}	%4100
  \rput(9.21,\nCBANDOddYOffset){\psframebox{21}}	%4120
  \rput(9.28,\nCBANDEvenYOffset){\psframebox{22}}	%4140
  \rput(9.35,\nCBANDOddYOffset){\psframebox{23}}	%4160
  \rput(9.42,\nCBANDEvenYOffset){\psframebox{24}}	%4180
}



{%K-Band satellite channels ref: "Telesat, Anik F3 Ku-Band Channel Centre Frequencies"
\small
% \newlength{\KBA}  \setlength{\KBA}{19.52in} %vertical position of kBand channels 1-16
% \newlength{\KBB}  \setlength{\KBB}{19.57in} %vertical position of kBand channels 17-32

  \psset{framesep=1pt,fillcolor=BrightGreen,fillstyle=solid,dotstyle=triangle,linecolor=BrightGreen}
% Upload channels 1-16
	\psdots(6.921,19.57)	% 14014750000.00Hz
	\psdots(6.952,19.57)	% 14045250000.00Hz
	\psdots(6.982,19.57)	% 14075750000.00Hz
	\psdots(7.013,19.57)	% 14106250000.00Hz
	\psdots(7.044,19.57)	% 14136750000.00Hz
	\psdots(7.074,19.57)	% 14167250000.00Hz
	\psdots(7.104,19.57)	% 14197750000.00Hz
	\psdots(7.135,19.57)	% 14228250000.00Hz
	\psdots(7.165,19.57)	% 14258750000.00Hz
	\psdots(7.195,19.57)	% 14289250000.00Hz
	\psdots(7.225,19.57)	% 14319750000.00Hz
	\psdots(7.255,19.57)	% 14350250000.00Hz
	\psdots(7.286,19.57)	% 14380750000.00Hz
	\psdots(7.315,19.57)	% 14411250000.00Hz
	\psdots(7.345,19.57)	% 14441750000.00Hz
	\psdots(7.375,19.57)	% 14472250000.00Hz
% Download channels 17-32
	\psdots(6.934,19.52)	% 14027750000.00Hz
	\psdots(6.965,19.52)	% 14058250000.00Hz
	\psdots(6.995,19.52)	% 14088750000.00Hz
	\psdots(7.026,19.52)	% 14119250000.00Hz
	\psdots(7.057,19.52)	% 14149750000.00Hz
	\psdots(7.087,19.52)	% 14180250000.00Hz
	\psdots(7.117,19.52)	% 14210750000.00Hz
	\psdots(7.148,19.52)	% 14241250000.00Hz
	\psdots(7.178,19.52)	% 14271750000.00Hz
	\psdots(7.208,19.52)	% 14302250000.00Hz
	\psdots(7.238,19.52)	% 14332750000.00Hz
	\psdots(7.268,19.52)	% 14363250000.00Hz
	\psdots(7.298,19.52)	% 14393750000.00Hz
	\psdots(7.328,19.52)	% 14424250000.00Hz
	\psdots(7.358,19.52)	% 14454750000.00Hz
	\psdots(7.388,19.52)	% 14485250000.00Hz

%Draw a little Ku satellite
\rput{15}(5.93,19.75){
  \psset{framesep=1pt,framearc=0,fillcolor=BrightGreen,fillstyle=solid,dotstyle=triangle,linewidth=.01in,linestyle=solid,linecolor=BrightGreen}
  \psline(-0.14,0)(0.14,0)
  \pscircle(0,0){0.035}
  \psframe(-.14,.02)(-.06,-.02)
  \psframe(.14,.02)(.06,-.02)
}
  \rput[t](5.93,19.68){\textcolor{BrightGreen}{Ku-band TV}}

%draw arrows showing signal in/out
{\psset{linecolor=BrightGreen,fillstyle=none,linewidth=1pt,linestyle=solid,linearc=.06}
 \psline{->}(7.148,19.60)(7.148,19.75)(6.08,19.75)
 \psline{->}(5.78,19.75)(4.6575,19.75)(4.6575,19.60)
}


% Download channels 1-16
	\psset{dotangle=180}
	\psdots(4.387,19.57)	% 11714750000.00Hz
	\psdots(4.423,19.57)	% 11745250000.00Hz
	\psdots(4.460,19.57)	% 11775750000.00Hz
	\psdots(4.497,19.57)	% 11806250000.00Hz
	\psdots(4.533,19.57)	% 11836750000.00Hz
	\psdots(4.569,19.57)	% 11867250000.00Hz
	\psdots(4.606,19.57)	% 11897750000.00Hz
	\psdots(4.642,19.57)	% 11928250000.00Hz
	\psdots(4.678,19.57)	% 11958750000.00Hz
	\psdots(4.714,19.57)	% 11989250000.00Hz
	\psdots(4.750,19.57)	% 12019750000.00Hz
	\psdots(4.786,19.57)	% 12050250000.00Hz
	\psdots(4.822,19.57)	% 12080750000.00Hz
	\psdots(4.857,19.57)	% 12111250000.00Hz
	\psdots(4.893,19.57)	% 12141750000.00Hz
	\psdots(4.928,19.57)	% 12172250000.00Hz
% Download channels 17-32
	\psdots(4.402,19.52)	% 11727750000.00Hz
	\psdots(4.439,19.52)	% 11758250000.00Hz
	\psdots(4.476,19.52)	% 11788750000.00Hz
	\psdots(4.512,19.52)	% 11819250000.00Hz
	\psdots(4.549,19.52)	% 11849750000.00Hz
	\psdots(4.585,19.52)	% 11880250000.00Hz
	\psdots(4.621,19.52)	% 11910750000.00Hz
	\psdots(4.657,19.52)	% 11941250000.00Hz
	\psdots(4.693,19.52)	% 11971750000.00Hz
	\psdots(4.729,19.52)	% 12002250000.00Hz
	\psdots(4.765,19.52)	% 12032750000.00Hz
	\psdots(4.801,19.52)	% 12063250000.00Hz
	\psdots(4.837,19.52)	% 12093750000.00Hz
	\psdots(4.872,19.52)	% 12124250000.00Hz
	\psdots(4.908,19.52)	% 12154750000.00Hz
	\psdots(4.943,19.52)	% 12185250000.00Hz
}



%water absorption line at 22GHz
% (3.5,20.05) -1.2x
  \rput(2.3,20.05){
  \psclip{\psframe[fillstyle=none,linestyle=none](0,0)(2.4,.3)}
    \pscurve[fillstyle=crosshatch,hatchcolor=Black,hatchangle=180,hatchwidth=0.5pt,hatchsep=1pt,fillcolor=blue,linecolor=Black,linestyle=solid]
	(0,.2)(.75,.15)(1,.02)(1.25,.15)(2,0.2)
    \rput(1,.2){\psframebox[framesep=1pt,framearc=0.2,fillstyle=solid, fillcolor=Black,linewidth=0pt,linestyle=none]{\textcolor{white}{Wasserabsorption 22GHz}}}
  \endpsclip}



{

	\tiny
	\psset{linewidth=0pt,linestyle=none,fillstyle=solid}

	% Derive colours from base color
	\colorlet{2GColor}{MobilePhoneColor!100!}
	\colorlet{3GColor}{MobilePhoneColor!70!}
	\colorlet{4GColor}{MobilePhoneColor!50!}
	\colorlet{5GColor}{MobilePhoneColor!20!}


	%%%%%%%%%%%%%%%%%%%%%%%%%%%%%%%%%%%%%%%%%%%%%%%%%%%%%%%%%%%%%%%%%%%%%%%%%%%%%
	% 2G GSM 900 and GSM 1800 (still in use as a fallback)
	%%%%%%%%%%%%%%%%%%%%%%%%%%%%%%%%%%%%%%%%%%%%%%%%%%%%%%%%%%%%%%%%%%%%%%%%%%%%%

	% Source: https://www.informationszentrum-mobilfunk.de/technik/funktionsweise/gsm
	{

		\psset{fillcolor=2GColor}

		\psframe(7.146,17.70)(7.538,17.85)\rput(7.34,17.78){GSM 900$\uparrow$}
		\psframe(7.844,17.70)(8.217,17.85)\rput(8.02,17.78){GSM 900$\downarrow$}

		\psframe(6.579,18.20)(7.186,18.35)\rput(6.88,18.28){GSM 1800$\uparrow$}
		\psframe(7.344,18.20)(7.919,18.35)\rput(7.63,18.28){GSM 1800$\downarrow$}

	}


	%%%%%%%%%%%%%%%%%%%%%%%%%%%%%%%%%%%%%%%%%%%%%%%%%%%%%%%%%%%%%%%%%%%%%%%%%%%%%
	% 3G UMTS (being phased out as of 2020)
	%%%%%%%%%%%%%%%%%%%%%%%%%%%%%%%%%%%%%%%%%%%%%%%%%%%%%%%%%%%%%%%%%%%%%%%%%%%%%

	% Source: https://www.informationszentrum-mobilfunk.de/technik/funktionsweise/umts
	% TODO: Verify
	{
		\psset{fillcolor=3GColor}
		\psframe(8.217,18.20)(9.80,18.35)\rput(8.9,18.28){3G / UMTS}\wrapuparrow{3GColor}{18.28}
		\psframe(0.00,18.55)(0.147,18.75)\wrapdnarrow{3GColor}{18.65}
	}


	%%%%%%%%%%%%%%%%%%%%%%%%%%%%%%%%%%%%%%%%%%%%%%%%%%%%%%%%%%%%%%%%%%%%%%%%%%%%%
	% 4G LTE
	%%%%%%%%%%%%%%%%%%%%%%%%%%%%%%%%%%%%%%%%%%%%%%%%%%%%%%%%%%%%%%%%%%%%%%%%%%%%%

	% https://www.lte-anbieter.info/ratgeber/frequenzen-lte.php
	% Nur mitteleuropäische Bänder
	{
		\psset{fillcolor=4GColor}
		% LTE Band 1 2110-2170 up, 1920-1980 down
		\psframe(9.551,18.05)(9.800,18.20)\rput(9.67,18.13){\shortstack{LTE 1$\uparrow$}}
		\wrapuparrow{4GColor}{18.13}
		\psframe(0.001,18.55)(0.147,18.75)
		\wrapdnarrow{4GColor}{18.65}
		\psframe(8.217,18.05)(8.652,18.20)\rput(8.43,18.13){\shortstack{LTE 1$\downarrow$}}
		% LTE Band 3 1710-1785 up, 1805-1880 down
		\psframe(6.579,18.05)(7.186,18.20)\rput(6.88,18.13){\shortstack{LTE Bd. 3$\uparrow$}}
		\psframe(7.344,18.05)(7.919,18.20)\rput(7.63,18.13){\shortstack{LTE Bd. 3$\downarrow$}}
		% LTE Band 7 2500-2570 up, 2620-2690 down
		\psframe(2.149,18.60)(2.539,18.70)\rput(2.34,18.65){\shortstack{LTE 7$\uparrow$}}
		\psframe(2.812,18.60)(3.185,18.70)\rput(3.00,18.65){\shortstack{LTE 7$\downarrow$}}
		% LTE Band 8 880-915 up, 925-960 down
		\psframe(6.987,17.55)(7.538,17.70)\rput(7.26,17.63){\shortstack{LTE Bd. 8$\uparrow$}}
		\psframe(7.692,17.55)(8.217,17.70)\rput(7.95,17.63){\shortstack{LTE Bd. 8$\downarrow$}}
		% LTE Band 20 832-862 up, 791-821 down
		% Überlappung zwischen 20 down und 28 up laut mehrerer Quellen
		\psframe(6.194,17.55)(6.695,17.75)\rput(6.44,17.65){\shortstack{LTE\\ Band 20$\uparrow$}}
		\psframe(5.479,17.55)(6.006,17.75)\rput(5.84,17.65){\shortstack{LTE\\ Bd. 20$\downarrow$}}
		% LTE Band 28 758-803 up, 703-748 down
		\psframe(4.877,17.55)(5.692,17.75)\rput(5.28,17.65){\shortstack{LTE\\ Band 28$\uparrow$}}
		\psframe(3.812,17.55)(4.689,17.75)\rput(4.25,17.65){\shortstack{LTE\\ Band 28$\downarrow$}}
		% LTE Band 32 1452-1496
		% Band 32 ist nur lokaler Zugang
		\psframe(4.267,18.05)(4.689,18.25)\rput(4.48,18.15){\shortstack{LTE\\ Band 32}}
	}

    %%%%%%%%%%%%%%%%%%%%%%%%%%%%%%%%%%%%%%%%%%%%%%%%%%%%%%%%%%%%%%%%%%%%%%%%%%%%%
	% 5G LTE
	%%%%%%%%%%%%%%%%%%%%%%%%%%%%%%%%%%%%%%%%%%%%%%%%%%%%%%%%%%%%%%%%%%%%%%%%%%%%%

	% 5G Bänder in Deutschland, Stand Ende 2020
	% Source: https://www.5g-anbieter.info/technik/5g-frequenzen.html
	{
		\psset{fillcolor=5GColor}
  		%n1 1920-1980 up,2210-2710 down
  		\psframe(8.217,18.05)(8.652,18.20)\rput(8.43,18.13){5G n1$\uparrow$}
  		\psframe(0.406,18.70)(3.289,18.80)\rput(1.85,18.75){5G n1$\downarrow$}
  		%n28 703-748 up, 758-803 down
  		\psframe(3.812,17.55)(4.689,17.75)\rput(4.25,17.65){5G n28$\uparrow$}
  		\psframe(3.812,17.55)(4.689,17.75)\rput(4.25,17.65){5G n28$\downarrow$}
  		%n78 3300-3800 TDD
  		\psframe(6.074,18.55)(8.069,18.75)\rput(7.07,18.65){5G n78}
  		% Pionierband n258 für local area (5G NR)
  		\psframe(4.873,20.05)(6.651,20.25)\rput(5.76,20.15){5G NR n258}
  	}



}

% Scientific spectrum use
{

\definecolor{IonosphericColor}{rgb}{0,0.2,1}
\tiny

%%%%%%%%%%%%%%%%%%%%%%%%%%%%%%%%%%%%%%%%%%%%%%%%%%%%%%%%%%%%%%%%%%%%%%%%%%%%%
% Ionosphärenforschung
%%%%%%%%%%%%%%%%%%%%%%%%%%%%%%%%%%%%%%%%%%%%%%%%%%%%%%%%%%%%%%%%%%%%%%%%%%%%%

\psset{linestyle=solid,linecolor=IonosphericColor,fillstyle=hlines,hatchangle=45,hatchcolor=IonosphericColor,fillcolor=IonosphericColor}

% EISCAT, source: https://eiscat.se/wp-content/uploads/2016/05/Technical.pdf
% Different freq ranges for RX and TX at different sites
% Aggregated to represent typical ranges

% Ionospheric Heater Tromsø 3.85-8 MHz
\psframe(8.589,13.40)(9.8,13.47)\wrapuparrow{IonosphericColor}{13.65}
\psframe(0, 14.01)(9.129,14.075)\rput(4.57,14.05){EISCAT Heater}\wrapdnarrow{IonosphericColor}{14.15}
% VHF Tromsø, Sodankylä and Kiruna % 214.3-234.7 MHz
\psframe(6.616,16.70)(7.901,16.85)\rput(7.26,16.77){EISCAT VHF}
% UHF Longyearbyen 485-515 MHz
\psframe(8.363,17.05)(9.212,17.25)\rput(8.79,17.15){EISCAT Longyearbyen}
% UHF Tromsø 921-933.5 MHz
\psframe(7.631,17.70)(7.821,17.85)\rput(7.71,17.775){\shortstack{EIS\\ CAT}}


%%%%%%%%%%%%%%%%%%%%%%%%%%%%%%%%%%%%%%%%%%%%%%%%%%%%%%%%%%%%%%%%%%%%%%%%%%%%%
% Radioastronomie
%%%%%%%%%%%%%%%%%%%%%%%%%%%%%%%%%%%%%%%%%%%%%%%%%%%%%%%%%%%%%%%%%%%%%%%%%%%%%

% 1420 MHz Wasserstofflinie
%Hydrogen line at 21cm (1.42758313333 GHz)
%Idea from Douglas Scott
%Info from http://www.drao-ofr.hia-iha.nrc-cnrc.gc.ca/outreach/skygazing/1997/sep1197.html
\blip{4.03,18}{H}


%%%%%%%%%%%%%%%%%%%%%%%%%%%%%%%%%%%%%%%%%%%%%%%%%%%%%%%%%%%%%%%%%%%%%%%%%%%%%
% X-Ray Crystallography / Röntgenkristallographie
%%%%%%%%%%%%%%%%%%%%%%%%%%%%%%%%%%%%%%%%%%%%%%%%%%%%%%%%%%%%%%%%%%%%%%%%%%%%%





% für niedrige Kanalkästchen erster Y-Wert -0,05 zweiter -0,15 Textpos. -0,1

}


  %O2 absorption line at 60GHz	http://www.wmo.ch/web/www/TEM/SG-RFC/Ch-5Final.doc
% (7.88,20.55) -1.2x
  \rput(6.68,20.55){
  \psclip{\psframe[fillstyle=none,linestyle=none](0,0)(2.4,.3)}
    \pscurve[fillstyle=crosshatch,hatchcolor=Black,hatchangle=180,hatchwidth=0.5pt,hatchsep=1pt,fillcolor=blue,linecolor=Black,linestyle=solid]
	(0,.2)(.75,.15)(1,.02)(1.25,.15)(2,0.2)
    \rput(1,.2){\psframebox[framesep=1pt,framearc=0.2,fillstyle=solid, fillcolor=Black,linewidth=0pt,linestyle=none]{\textcolor{white}{$O_2$ absorption 60GHz}}}
  \endpsclip}







  % Ham bands need to be included after the microwave bands in order to be drawn atop
  % In Deutschland freigegebene Amateurfunkbänder
%
% Bei Bändern mit sehr unterschiedlichen Nutzungseinschränkungen werden die äußersten Bandgrenzen genutzt
%
% Quelle:  https://de.wikipedia.org/wiki/Amateurfunkband
% Prüfung: https://www.bundesnetzagentur.de/DE/Sachgebiete/Telekommunikation/Unternehmen_Institutionen/Frequenzen/Grundlagen/Frequenzplan/frequenzplan-node.html
% Stand: November 2020

{

\psset{linestyle=none,fillcolor=green,fillstyle=solid}

% 2,2 km 135,7-137.8 kHz
\psframe(0.491,11.55)(0.501,11.75)\rput(0.51,11.65){2,2\ km}

% 630 Meter 472-479 kHz
\psframe(8.315,12.05)(8.523,12.25)\rput(8.42,12.15){630\ m}

% 160 Meter 1.8-2.0MHz
\psframe(7.718,13.05)(9.129,13.25)\rput(8.42,13.15){160\ m}

% 80 Meter 3.5-3.8MHz
\psframe(7.241,13.55)(8.404,13.75)\rput(7.82,13.65){80\ m}

% 60 Meter "Tropenband"
\psframe(3.348,14.05)(3.486,14.25)\rput(3.42,14.15){\tiny 60m}

% 40 Meter 7.0-7.2MHz
\psframe(7.241,14.05)(7.640,14.25)\rput(7.44,14.15){40\ m}

% 30 Meter 10-10.150MHz
\psframe(2.62,14.55)(2.69,14.75)\rput(2.66,14.65){30m}

% 20 Meter 14.0-14.35MHz
\psframe(7.24,14.55)(7.59,14.75)\rput(7.42,14.65){20\ m}

% 17 Meter 18068-18168 kHz
\psframe(1.05,15.05)(1.13,15.25)\rput(1.09,15.2){17m}

% 15 Meter 21.0-21.45MHz
\psframe(3.17,15.05)(3.47,15.25)\rput(3.32,15.15){15\ m}

% 12 Meter 24890-24990 kHz
\psframe(5.58,15.05)(5.63,15.25)\rput(5.61,15.15){12\ m}

% 10 Meter 28.0-29.7MHz
\psframe(7.241,15.05)(8.075,15.25)\rput(7.66,15.15){10\ m}

% 6 Meter 50-54MHz
\psframe(5.639,15.55)(6.194,15.75)\rput(5.92,15.65){6\ m}

% 2 Meter 144-146MHz
\psframe(0.995,16.55)(1.190,16.75)\rput(1.09,16.65){2\ m}

% 70 cm 430-440 MHz
\psframe(6.662,17.05)(6.987,17.25)\rput(6.82,17.15){70\ cm}

% 23 cm  1240-1300 MHz
\psframe(2.035,18.05)(2.703,18.25)\rput(2.37,18.15){23\ cm}

% 13 cm 2320-2450 MHz
\psframe(1.092,18.55)(1.863,18.75)\rput(1.48,18.65){13\ cm}

% 9 cm 3,4-3,475 GHz
\psframe(6.496,18.55)(6.805,18.75)\rput(6.65,18.65){9\ cm}

% 6 cm 5,65-5,85 GHz
\psframe(3.877,19.05)(4.369,19.25)\rput(4.12,19.15){6\ cm}

% 3 cm 10-10,5 GHz
\psframe(2.149,19.55)(2.839,19.75)\rput(2.49,19.65){3\ cm}

% 1,2 cm 24-24,25 GHz
\psframe(4.727,20.05)(4.873,20.25)\rput(4.80,20.15){1,2\ cm}

% 6 mm 47-47,2 GHz
\psframe(4.429,20.55)(4.489,20.75)\rput(4.46,20.65){6\ mm}

% 4 mm 76-81,5 GHz
\psframe(1.424,21.05)(2.412,21.25)\rput(1.92,21.15){4\ mm}

% 2,5 mm 122,25-123 GHz
\psframe(8.144,21.05)(8.231,21.25)\rput(8.19,21.15){2,5\ mm}

% 2 mm 134-141 GHz
\psframe(9.442,21.05)(9.755,21.25)\rput(4.90,21.40){2\ mm}\wrapuparrow{green}{21.2}
\psframe(0.006,21.55)(0.362,21.75)\rput(0.18,21.65){2\ mm}\wrapdnarrow{green}{21.6}

% 1,2 mm 241-250 GHz
\psframe(7.940,21.55)(8.459,21.75)\rput(8.20,21.65){1,2\ mm}

}


  %O2 absorption line at 118.75GHz	http://www.wmo.ch/web/www/TEM/SG-RFC/Ch-5Final.doc
% (7.73,21.05) -1.2x
  \rput(6.53,21.05){
  \psclip{\psframe[fillstyle=none,linestyle=none](0,0)(2.4,.3)}
    \pscurve[fillstyle=crosshatch,hatchcolor=Black,hatchangle=180,hatchwidth=0.5pt,hatchsep=1pt,fillcolor=blue,linecolor=Black,linestyle=solid]
	(0,.2)(.75,.15)(1,.02)(1.25,.15)(2,0.2)
    \rput(1,.2){\psframebox[framesep=1pt,framearc=0.2,fillstyle=solid, fillcolor=Black,linewidth=0pt,linestyle=none]{\textcolor{white}{$O_2$ absorption 118.75GHz}}}
  \endpsclip}




  %water absorption line at 183GHz
% (4.05,21.55) -1.2x
  \rput(2.85,21.55){
  \psclip{\psframe[fillstyle=none,linestyle=none](0,0)(2.4,.3)}
    \pscurve[fillstyle=crosshatch,hatchcolor=Black,hatchangle=180,hatchwidth=0.5pt,hatchsep=1pt,fillcolor=blue,linecolor=Black,linestyle=solid]
	(0,.2)(.75,.15)(1,.02)(1.25,.15)(2,0.2)
    \rput(1,.2){\psframebox[framesep=1pt,framearc=0.2,fillstyle=solid, fillcolor=Black,linewidth=0pt,linestyle=none]{\textcolor{white}{Water absorption 183GHz}}}
  \endpsclip}


  % Microwave u mm band (300 - 3000 GHz)
%  \definecolor{Fill}{rgb}{1.0,0.1,0.67}\psset{hatchcolor=Fill}
%  \psframe(1.24,22.05)(9.80,22.25)\wrapuparrow{Fill}{22.15}
%  \psframe(0.00,22.55)(9.80,22.75)\wrapbtarrow{Fill}{22.65}
%  \psframe(0.00,23.05)(9.80,23.25)\wrapbtarrow{Fill}{23.15}
%  \psframe(0.00,23.55)(4.39,23.75)\wrapdnarrow{Fill}{23.65}
%  \rput(1.24,22.15){\textcolor{white}{300GHz}}
%  \rput(8,23.15){\psframebox[fillstyle=solid,fillcolor=Fill,framesep=2pt]{\textcolor{white}{Microwave \micro mm-band}}}





% Er - Erbium-Doped Fiber Amplifier (EDFA) Thanks to E. Alan Dowdell for this information and his papers
  \fiberoptics{4.52,26.55}{EDFA} % C-band (1530 - 1565nm)
  \fiberoptics{4.14,26.55}{EDFA} % L-band (1570-1610nm)
  \fiberoptics{3.19,27.05}{Datacom} % <300m distances (850nm)
  \fiberoptics{6.87,26.55}{Telecom} %  for shorter distances, (1310nm)



  %Land to submarine communications
  %http://www.vlf.it/trond2/backgroud.html contains some description
  %http://www.vlf.it/trond2/25-30khz.html contains listing of more stations
  %http://www.random-abstract.com/radio/ contains listing of more stations
  %http://www.fas.org/nuke/guide/usa/c3i/vlf.htm contains listing of more stations
  %\submarine{3.10,9.6}{10.2kHz} %now defunct according to Poul-Henning Kamp
  %\submarine{5.40,9.6}{12kHz} %now defunct according to Poul-Henning Kamp
  \submarine{7.17,9.65}{13.6kHz}
  \submarine{8.27,9.65}{14.7kHz}
  \submarine{9.02,9.55}{15.5kHz}
  %\submarine{9.12,9.6}{15.62kHz}% too close to others to show

  \submarine{1.17,10.15}{17.8kHz}

  \submarine{1.79,10.15}{18.6kHz}

  \submarine{3.78,10.15}{21.4kHz}
  \submarine{4.36,10.15}{22.3kHz}
  %\submarine{4.39,10.1}{22.35kHz}% too close to others to show


 %24.8kHz, 25.0kHz, 25.1kHz, 25.2kHz, 25.3kHz, 25.5kHz, 25.6kHz, 25.7kHz, 25.8kHz, 26.1kHz, 26.2kHz, 26.3kHz, 26.5kHz, 26.6kHz, 26.7kHz, 26.8kHz, 26.9kHz 27.7kHz, 27.9kHz,
 %27.0kHz, 27.2kHz, 27.3kHz, 27.5kHz, 27.6kHz, 28.0kHz, 28.5kHz, 28.6kHz, 29.0kHz, 29.3kHz, 29.4kHz, 29.5kHz, 29.6kHz, 30.0kHz

  \psline[showpoints=true,linecolor=green,linestyle=solid,linewidth=1pt,dotstyle=*](5.40,10.15)(5.86,10.15)
(5.97,10.15)(6.03,10.15)(6.09,10.15)(6.14,10.15)
(6.25,10.15)(6.31,10.15)(6.36,10.15)(6.42,10.15)
(6.58,10.15)(6.64,10.15)(6.69,10.15)(6.80,10.15)
(6.85,10.15)(6.90,10.15)(6.96,10.15)(7.01,10.15)
(7.06,10.15)(7.17,10.15)(7.22,10.15)(7.32,10.15)
(7.37,10.15)(7.42,10.15)(7.53,10.15)(7.58,10.15)
(7.83,10.15)(7.88,10.15)(8.07,10.15)(8.22,10.15)
(8.27,10.15)(8.31,10.15)(8.36,10.15)(8.55,10.15)

  \submarine{5.40,10.15}{24kHz}%NAA from Maine
  \submarine{8.55,10.15}{30.0kHz}

  \submarine{3.08,10.6}{40.75kHz}

  \submarine{2.43,6.1}{76Hz}

{
%GPS submitted by Poul-Henning Kamp
%Each frequency is +/- 15MHz  according to details at http://www.edu-observatory.org/gps/BostonSection.ppt
%
\psset{fillstyle=solid, fillcolor=yellow,linecolor=yellow,linewidth=1pt,linestyle=solid,framearc=0.4, framesep=0}
% GPS-L1: 1575.4200 MHz (Civil Frequency  Coarse Acquisition)
\psline(5.42,18.20)(5.42,18)
\rput(5.42,18.15){\psframebox[boxsep=false]{\parbox{0.25in}{\centering GPS\\\tiny Civil}}}

% GPS-L2: 1227.60 MHz (Military Frequency  	(encrypted) Available 2003)
\psline(1.89,18.20)(1.89,18)
\rput(1.89,18.15){\psframebox[boxsep=false]{\parbox{0.25in}{\centering GPS\\\tiny Mil.}}}

% GPS-L3: 1381.0500 MHz (Nuclear Burst Detection  	NUDET - Nuclear Detonation)
\psline(3.56,18.20)(3.56,18)
\rput(3.56,18.15){\psframebox[boxsep=false]{\parbox{0.27in}{\centering GPS\\\tiny NUDET}}}

% GPS-L4: 1841.4000 MHz (Ionospheric correction  	Proposed)
\psline(7.63,18.20)(7.63,18)
\rput(7.63,18.15){\psframebox[boxsep=false]{\parbox{0.25in}{\centering GPS\\\tiny Ion.}}}

% GPS-L5: 1176.45 MHz (Proposed Frequency for 2008)
\psline(1.29,18.20)(1.29,18)
\rput(1.29,18.15){\psframebox[boxsep=false]{\parbox{0.25in}{\centering GPS\\\tiny SoL}}}

% GPS 2227.5000MHz Spacecraft Telemetry
\psline(0.52,18.20)(0.52,18)
\rput(0.52,18.15){\psframebox[boxsep=false]{\parbox{0.25in}{\centering GPS\\\tiny Space}}}
}


% % Cs-133 hyperfine transistion at 9,192,631,770Hz, the SI-definition of a second. Thanks to Poul-Henning Kamp
\rput{0}(0.96,19.75){\psframebox[framesep=1pt,framearc=0.2,fillstyle=solid, fillcolor=Black,linewidth=0pt,linestyle=none]{\textcolor{white}{Cs-133 9,192,631,770Hz SI-time standard}}}
\rput(0.96,19.57){\timestandard}





  %TV horizontal refresh
  \rput(9.24,9.7){
\psframe[cornersize=relative,linecolor=white, linestyle=solid, linewidth=0.8pt,fillstyle=solid,framearc=.25,fillcolor=red,linearc=.25](-.1,-.1)(.1,.1)
\psline[linecolor=white,linestyle=solid,linewidth=1pt]{<->}(-.1,0)(.1,0)
  }



  %Submarine ELF transmission
  %\definecolor{BoxColor}{rgb}{.7,1,.7}
  %\psline[linecolor=BoxColor](2.43,6.0)(2.43,6.3)
  %\psframe[framearc=0.25, fillstyle=solid, fillcolor=BoxColor,linewidth=0pt,linestyle=none](1.96,6.05)(2.90,6.25)
  %\rput(2.43,6.2){Submarine}
  %\rput(2.43,6.1){ELF Transmissions}



}% End of tilt
}% End of rotate


  % frequency band labels
  %  - printed on top of other graphics
  %  - not rotated and tilted like the rest of the chart

  %y position = xpos_xvalue/9.8 * 0.5" + nearest xpos_yvalue
  %Ex. xpos results for  750e12 937e12   =    (3.72,27.55)(6.62,27.75)   (5.17,27.65)
  %    3.72/9.8 * 0.5 + 27.5 = 27.690
  %\psframe[fillstyle=solid, fillcolor=BoxColor, linecolor=BoxColor](-0.1,27.690)(9.9,27.838)

  \definecolor{RangesText}{rgb}{1.0,0.7,0.7}


  \definecolor{DrawingColor}{rgb}{1.0,0.7,0.7}
  \psset{linecolor=red,framearc=0, fillstyle=solid,linecolor=DrawingColor,gradangle=270,gradmidpoint=1.0}

  % ULF	Ultra Low Frequency  3 - 30Hz  source  http://www.haarp.alaska.edu/haarp/elf.html
  \psline{<->}(\EMRPosition,3.792)(\EMRPosition,5.453)
  \rput{90}(\EMRPosition,4.623){\psframebox[fillcolor=LightRange]{\textcolor{RangesText}{ULF Ultra Low Frequency}}}
  \rput[t]{90}(\EMRPositionC,3.792){\psframebox[linestyle=none, fillstyle=gradient,gradbegin=LightRange,gradend=Black]{\textcolor{RangesText}{3Hz}}}

  %ELF Extremely Low Frequency  30 - 3kHz
  \psline{<->}(\EMRPosition,5.453)(\EMRPosition,8.776)
  \rput{90}(\EMRPosition,7.115){\psframebox[fillcolor=DarkRange]{\textcolor{RangesText}{ELF Extremely Low Frequency}}}
  \rput[t]{90}(\EMRPositionC,5.453){\psframebox[linestyle=none, fillstyle=gradient,gradbegin=DarkRange,gradend=LightRange]{\textcolor{RangesText}{30Hz}}}

  % VLF	Very Low Frequency	100km - 10km	3kHz - 30kHz
  \psline{<->}(\EMRPosition,8.776)(\EMRPosition,10.436)
  \rput{90}(\EMRPosition,9.577){\psframebox[fillcolor=LightRange]{\textcolor{RangesText}{VLF Very Low Frequency}}}
  \rput[t]{90}(\EMRPositionC,8.776){\psframebox[linestyle=none, fillstyle=gradient,gradbegin=LightRange,gradend=DarkRange]{\textcolor{RangesText}{3kHz}}}

  % LF 	Low Frequency		10km - 1km	30kHz - 300kHz
  \psline{<->}(\EMRPosition,10.436)(\EMRPosition,12.097)
  \rput{90}(\EMRPosition,11.245){\psframebox[fillcolor=DarkRange]{\textcolor{RangesText}{LF Low Frequency}}}
  \rput[t]{90}(\EMRPositionC,10.436){\psframebox[linestyle=none, fillstyle=gradient,gradbegin=DarkRange,gradend=LightRange]{\textcolor{RangesText}{30kHz}}}

  % MF 	Medium frequency 	1km - 100m	300-3000 kHz
  \psline{<->}(\EMRPosition,12.097)(\EMRPosition,13.758)
  \rput{90}(\EMRPosition,12.913){\psframebox[fillcolor=LightRange]{\textcolor{RangesText}{MF Medium Frequency}}}
  \rput[t]{90}(\EMRPositionC,12.097){\psframebox[linestyle=none, fillstyle=gradient,gradbegin=LightRange,gradend=DarkRange]{\textcolor{RangesText}{300kHz}}}

  % HF 	High frequency 		100m - 10m	3-30 MHz
  \psline{<->}(\EMRPosition,13.758)(\EMRPosition,15.419)
  \rput{90}(\EMRPosition,14.561){\psframebox[fillcolor=DarkRange]{\textcolor{RangesText}{HF High Frequency}}}
  \rput[t]{90}(\EMRPositionC,13.758){\psframebox[linestyle=none, fillstyle=gradient,gradbegin=DarkRange,gradend=LightRange]{\textcolor{RangesText}{3MHz}}}

  % VHF	Very High Frequency	10m - 1m	30 - 300 MHz
  \psline{<->}(\EMRPosition,15.419)(\EMRPosition,17.080)
  \rput{90}(\EMRPosition,16.229){\psframebox[fillcolor=LightRange]{\textcolor{RangesText}{VHF Very High Frequency}}}
  \rput[t]{90}(\EMRPositionC,15.419){\psframebox[linestyle=none, fillstyle=gradient,gradbegin=LightRange,gradend=DarkRange]{\textcolor{RangesText}{30MHz}}}

  % UHF	Ultra High Frequency	1m - 10cm	300MHz - 3GHz
  \psline{<->}(\EMRPosition,17.080)(\EMRPosition,18.741)
  \rput{90}(\EMRPosition,17.897){\psframebox[fillcolor=DarkRange]{\textcolor{RangesText}{UHF Ultra High Frequency}}}
  \rput[t]{90}(\EMRPositionC,17.080){\psframebox[linestyle=none, fillstyle=gradient,gradbegin=DarkRange,gradend=LightRange]{\textcolor{RangesText}{300MHz}}}

  % SHF Super High Frequency	10cm - 1cm	3GHz - 30GHz
  \psline{<->}(\EMRPosition,18.741)(\EMRPosition,20.402)
  \rput{90}(\EMRPosition,19.545){\psframebox[fillcolor=LightRange]{\textcolor{RangesText}{SHF Super High Frequency}}}
  \rput[t]{90}(\EMRPositionC,18.741){\psframebox[linestyle=none, fillstyle=gradient,gradbegin=LightRange,gradend=DarkRange]{\textcolor{RangesText}{3GHz}}}

  % EHF Extremely High Frequency 1cm - 1mm	30GHz - 300GHz
  \psline{<->}(\EMRPosition,20.402)(\EMRPosition,22.063)
  \rput{90}(\EMRPosition,21.214){\psframebox[fillcolor=DarkRange]{\footnotesize\textcolor{RangesText}{EHF Extremely High Frequency}}}
  \rput[t]{90}(\EMRPositionC,20.402){\psframebox[linestyle=none, fillstyle=gradient,gradbegin=DarkRange,gradend=LightRange]{\textcolor{RangesText}{30GHz}}}
  \rput[t]{90}(\EMRPositionC,22.063){\psframebox[linestyle=none, fillstyle=gradient,gradbegin=Black,gradend=DarkRange]{\textcolor{RangesText}{300GHz}}}


  % Microwave mm-band (40-300 GHz)
  \psline{<->}(\EMRPositionD,20.610)(\EMRPositionD,22.063)
  \rput{90}(\EMRPositionD,21.34){\psframebox[fillcolor=DarkRange]{\footnotesize\textcolor{RangesText}{Microwave mm-band}}}

  % Microwave u mm band (300 - 3000 GHz)(Terahertz 1mm - 100umm)
  \psline{<->}(\EMRPositionD,22.063)(\EMRPositionD,23.724)
	\rput{90}(\EMRPositionD,22.89){%
		\psframebox[fillcolor=DarkRange]{%
			\parbox{1.3in}{%
				\centering
				\footnotesize \textcolor{RangesText}{Microwave\ } \SI[color=RangesText]{}{\micro\milli\meter-band}\\\textcolor{RangesText}{Terahertz rays}
			}%
		}%
	}%
  \rput[t]{90}(\EMRPositionC,23.724){\psframebox[linestyle=none, fillstyle=solid,fillcolor=Black]{\textcolor{RangesText}{3THz}}}


  % Draw Ultraviolet labels for NUV, MUV, FUV, EUV, VUV
  \rput[t]{90}(\EMRPositionC,27.707){\psframebox[linestyle=none, fillstyle=gradient,gradbegin=DarkRange,gradend=Black]{\textcolor{RangesText}{\SI{400}{\nano\meter}}}}
  \rput{90}(\EMRPositionD,27.95){\psframebox[linestyle=none, fillstyle=gradient,gradbegin=LightRange,gradend=DarkRange]{\textcolor{RangesText}{\SI{300}{\nano\meter}}}}%y was 27.916
  \rput[t]{90}(\EMRPositionC,28.207){\psframebox[linestyle=none, fillstyle=gradient,gradbegin=DarkRange,gradend=LightRange]{\textcolor{RangesText}{\SI{200}{\nano\meter}}}}
  \rput[t]{90}(\EMRPositionC,28.705){\psframebox[linestyle=none, fillstyle=gradient,gradbegin=LightRange,gradend=DarkRange]{\textcolor{RangesText}{\SI{100}{\nano\meter}}}}
  \rput[t]{90}(\EMRPositionC,30.368){\psframebox[linestyle=none, fillstyle=gradient,gradbegin=Black,gradend=LightRange]{\textcolor{RangesText}{\SI{10}{\nano\meter}}}}
  \psline{<->}(\EMRPosition,27.707)(\EMRPosition,27.916)% NUV arrow
  \psline{<->}(\EMRPosition,27.916)(\EMRPosition,28.207)% MUV arrow
  \psline{<->}(\EMRPosition,28.207)(\EMRPosition,28.705)% FUV arrow
  \psline{<->}(\EMRPosition,28.705)(\EMRPosition,30.368)% EUV arrow
  \psline{<->}(\EMRPositionD,28.207)(\EMRPositionD,30.368)% VUV arrow
  \rput(\EMRPosition,27.812){\psframebox[fillcolor=DarkRange]{\textcolor{RangesText}{NUV}}}
  \rput(\EMRPosition,28.062){\psframebox[fillcolor=LightRange]{\textcolor{RangesText}{MUV}}}
  \rput(\EMRPosition,28.456){\psframebox[fillcolor=DarkRange]{\textcolor{RangesText}{FUV}}}
  \rput(\EMRPosition,29.75){\psframebox[fillcolor=LightRange]{\textcolor{RangesText}{EUV}}}
  \rput{90}(\EMRPositionD,29.4){\psframebox[fillcolor=LightRange]{\textcolor{RangesText}{VUV}}}



  %Long-wave radio  5kHz-540kHz  http://www.dxing.com/lw.htm
  % Includes traditional AM radio region. These frequencies can travel long distances by multiple
  %reflections between the surface of the earth and its ionosphere.
  \psline[linecolor=RangesText]{<->}(\EMRPositionD,9.144)(\EMRPositionD,12.521)
  \rput{90}(\EMRPositionD,10.83){\psframebox[fillstyle=gradient,gradbegin=DarkRange,gradend=LightRange,gradangle=270,gradmidpoint=1]{\textcolor{RangesText}{Langwellenradio}}}

  %Short-wave radio 1700kHz-30MHz  http://www.dxing.com/swlintro.htm
  % Used for TV, FM, and other communication purposes. Generally travels only relatively
  %short distances because the ionosphere is transparent to it.
  \psline[linecolor=RangesText]{<->}(\EMRPositionD,13.349)(\EMRPositionD,15.419)
  \rput{90}(\EMRPositionD,14.38){\psframebox[fillstyle=solid,fillcolor=DarkRange]{\textcolor{RangesText}{Kurzwellenradio}}}


%%%%%%%%%%%%%%%%%%%%%%%%%%%%%%%%%%%%%%%%%%%%%%%%%%%%%%%%%%%%%%%%%%%%%%%%%%%%%
% Pressure waves, bottom right
%%%%%%%%%%%%%%%%%%%%%%%%%%%%%%%%%%%%%%%%%%%%%%%%%%%%%%%%%%%%%%%%%%%%%%%%%%%%%

  %Audio range
  \psset{framearc=0}
  \definecolor{AudioLineColor}{rgb}{1,1,1}
  \definecolor{UltrasonicAudioColor}{rgb}{0.2,0.2,0.6}%B was 0.4
  \definecolor{SubSonicColor}{rgb}{0.2,0.2,0.6}%B was 0.4
  \definecolor{DarkBackground}{rgb}{0.2,0.2,0.2}
  \definecolor{LightBackground}{rgb}{0.4,0.4,0.4}
  \definecolor{BoxColor}{rgb}{0.2,0.2,0.2}
  \newlength{\LowerAudioRange}		\setlength{\LowerAudioRange}{5.161in} %20Hz
  \newlength{\LowerSubSonicRange}	\setlength{\LowerSubSonicRange}{\LowerAudioRange-1.5in}
  \newlength{\UpperAudioRange}		\setlength{\UpperAudioRange}{10.144in} %20kHz
  \newlength{\UpperUltrasonicRange}	\setlength{\UpperUltrasonicRange}{\UpperAudioRange+1in}

  % Fill backround of these with RGB colors so that fades appear smooth
  \psframe[fillstyle=gradient,
	gradangle=0,
	gradbegin=RGBBlackBegin,
	gradend=RGBBlackEnd,
	gradmidpoint=1.0,
	linewidth=0pt,
	linestyle=none](\AudioPositionC,\UpperUltrasonicRange)(\AudioPositionD,0)

  % Ultrasonic audio range  20kHz - infinitely large
  \psframe[fillstyle=gradient,
  	gradangle=0,
	gradbegin=UltrasonicAudioColor,
	gradend=black,
	gradmidpoint=0,
	linewidth=0pt,
	linestyle=none](\AudioPositionC,\UpperAudioRange)(\AudioPositionD,\UpperUltrasonicRange)

  % Human audio range  20Hz - 20kHz
  \psframe[fillstyle=solid,
	fillcolor=HumanAudioColor,
	linewidth=0pt,
	linestyle=none](\AudioPositionC,\LowerAudioRange)(\AudioPositionD,\UpperAudioRange)

  % Subsonic audio range  infinitely small - 20Hz
  \psframe[fillstyle=gradient,
	gradangle=0,
	gradbegin=SubSonicColor,
	gradend=black,
	gradmidpoint=1.0,
	linewidth=0pt,
	linestyle=none](\AudioPositionC,\LowerAudioRange)(\AudioPositionD,\LowerSubSonicRange)

  %Piano keys starting place at 27.5Hz - 0.071"/2
  \rput{0}(\AudioPositionE,5.355){\psframebox[fillstyle=none,linewidth=0pt,linestyle=none]
  	{\parbox[t]{.001in}{\input{tex/piano.tex}}}}

%Heart Beats
% More information can be found here http://hypertextbook.com/facts/1998/ArsheAhmed.shtml
% Infant 120 bpm = 2.0Hz
% Children 90 bpm = 1.5Hz
% Adult 70 bpm = 1.16Hz
\psframe[fillstyle=gradient,
	gradangle=0,
	gradbegin=black,
	gradend=red,
	gradmidpoint=1.0,
	linewidth=0pt,
	linestyle=none](\AudioPositionC,3.500)(\AudioPositionD,3.600)
\psframe[fillstyle=solid,
	linewidth=0pt,
	fillcolor=red,
	linestyle=none](\AudioPositionC,3.107)(\AudioPositionD,3.500)
\psframe[fillstyle=gradient,
	gradangle=0,
	gradbegin=red,
	gradend=black,
	gradmidpoint=1.0,
	linewidth=0pt,
	linestyle=none](\AudioPositionC,3.007)(\AudioPositionD,3.107)
\rput(\AudioPositionH,3.29){\begin{minipage}[t]{0.6in}{\begin{center}\textcolor{white}{Heart beats}\end{center}}\end{minipage}}


%Ocean waves thanks to Ward Cunningham
% More information can be found here http://oceanworld.tamu.edu/resources/ocng_textbook/chapter16/chapter16_04.htm
% Starting at .05Hz
% Peak at .07Hz
% Ending at ,3Hz
\psframe[fillstyle=gradient,
	gradangle=0,
	gradbegin=blue,
	gradend=black,
	gradmidpoint=1.0,
	linewidth=0pt,
	linestyle=none](\AudioPositionC,.839)(\AudioPositionD,1.082)
\psframe[fillstyle=gradient,
	gradangle=0,
	gradbegin=black,
	gradend=blue,
	gradmidpoint=1.0,
	linewidth=0pt,
	linestyle=none](\AudioPositionC,1.082)(\AudioPositionD,2.132)
\rput(\AudioPositionH,1.12){\begin{minipage}[t]{0.6in}{\begin{center}\textcolor{white}{Ocean waves}\end{center}}\end{minipage}}

\psframe[fillstyle=solid,
	linewidth=0pt,
	fillcolor=FColor,
	linestyle=none](\AudioPositionC,.36)(\AudioPositionD,0)
\rput(\AudioPositionH,0.19){\begin{minipage}[t]{0.6in}{\begin{center}\textcolor{black}{Pressure waves}\end{center}}\end{minipage}}
% \rput(\AudioPositionE,.19){\textcolor{black}{Mechanical}}
% \rput(\AudioPositionE,.07){\textcolor{black}{Waves}}

%Audio left side arrows
\psline[linecolor=AudioLineColor]{>-}(\AudioPositionC,\UpperAudioRange)(\AudioPositionC,\UpperUltrasonicRange) %Ultrasonic arrow
\psline[linecolor=AudioLineColor]{<->}(\AudioPositionC,\LowerAudioRange)(\AudioPositionC,\UpperAudioRange) %Audio range arrow
\psline[linecolor=AudioLineColor]{-<}(\AudioPositionC,0)(\AudioPositionC,\LowerAudioRange) %Subsonic arrow
%Audio left side labels
\rput{90}(\AudioPositionC,7.64){%
	\psframebox[framearc=0.25,fillstyle=solid,fillcolor=BoxColor,linecolor=AudioLineColor]{\textcolor{AudioLineColor}{Menschl.Hörbereich}}}
\rput{90}(\AudioPositionC,10.644){%
	\psframebox[framearc=0.25, fillstyle=solid,fillcolor=BoxColor, linecolor=AudioLineColor]{\textcolor{AudioLineColor}{Ultraschall}}}
\rput{90}(\AudioPositionC,4.4){%
	\psframebox[framearc=0.25, fillstyle=solid,fillcolor=BoxColor, linecolor=AudioLineColor]{\textcolor{AudioLineColor}{Infraschall}}}



%\psline[linecolor=white,linewidth=0.7pt](\AudioPositionC,\UpperUltrasonicRange)(\AudioPositionC,0) %Left border line
% \psline[linecolor=white,linewidth=0.7pt]{->}(\AudioPositionC,\LowerSubSonicRange)(\AudioPositionC,0) %Left border line
\psline[linecolor=white,linewidth=0.7pt](\AudioPositionD,\UpperUltrasonicRange)(\AudioPositionD,0) %Right border line


  %Label "one cycle per second" so people can easily see the starting point (1Hz) of the frequency marker on the poster
  % Leave room on the right side for heartbeats.
%   \rput(-.15,3.08){
%   	\psline[linecolor=yellow,linewidth=2pt]{->}(-.5,0)(0,0)
%   	\rput[r]{0}(-.2,0){
%   		\psframebox[fillstyle=solid,fillcolor=yellow,linewidth=1pt,linecolor=yellow,framearc=0.25]{
% 			\parbox[t]{.6in}{One Cycle Per Second}
% 		}
%   	}
%   }
\rput(-.85,2.93){
	\psline[linecolor=yellow,linewidth=2pt]{->}(0,0)(0.65,0)%points to grid
	\rput[r]{0}(0,0){
		\pscircle[linestyle=none, fillcolor=FColor, fillstyle=solid](0,0){0.44}
		\psarc[linecolor=black, linewidth=2pt, linestyle=solid,fillstyle=none]{<-<}(0,0){0.37in}{90}{85}
		}
	\rput(0,0){\parbox[t]{.6in}{\centering Eine Schwingung pro Sekunde}}
}


%For more info, see: http://www.arpansa.gov.au/basics/ion_nonion.htm  or http://hyperphysics.phy-astr.gsu.edu/hbase/nuclear/radrisk.html
%Ionizing radiation (>10eV) has enough energy to strip electrons from an atom or, in the case of very high-energy radiation, break up the nucleus of the atom. This effect can cause damage to living tissue.
%Must be moved over a little to the right to avoid hitting the suffixes
\psline[linewidth=0.14in,linecolor=IonizingYellow]{->}(1.11,28.551)(1.11,34.5)
\rput[l]{90}(1.12,28.6){Ionisierende Strahlung, schädlich für lebende Organismen}

  %Cosmic Background Radiation
  \rput{0}(10.2,20.96){\input{tex/cmb.tex}}


